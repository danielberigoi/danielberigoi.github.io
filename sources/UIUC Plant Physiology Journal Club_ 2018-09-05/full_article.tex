\documentclass[10pt]{article}

\usepackage{fullpage}
\usepackage{setspace}
\usepackage{parskip}
\usepackage{titlesec}
\usepackage{placeins}
\usepackage{xcolor}
\usepackage{breakcites}
\usepackage{lineno}





\PassOptionsToPackage{hyphens}{url}
\usepackage[colorlinks = true,
            linkcolor = blue,
            urlcolor  = blue,
            citecolor = blue,
            anchorcolor = blue]{hyperref}
\usepackage{etoolbox}
\makeatletter
\patchcmd\@combinedblfloats{\box\@outputbox}{\unvbox\@outputbox}{}{%
  \errmessage{\noexpand\@combinedblfloats could not be patched}%
}%
\makeatother


\usepackage[round]{natbib}
\let\cite\citep




\renewenvironment{abstract}
  {{\bfseries\noindent{\abstractname}\par\nobreak}\footnotesize}
  {\bigskip}

\renewenvironment{quote}
  {\begin{tabular}{|p{13cm}}}
  {\end{tabular}}

\titlespacing{\section}{0pt}{*3}{*1}
\titlespacing{\subsection}{0pt}{*2}{*0.5}
\titlespacing{\subsubsection}{0pt}{*1.5}{0pt}


\usepackage{authblk}


\usepackage{graphicx}
\usepackage[space]{grffile}
\usepackage{latexsym}
\usepackage{textcomp}
\usepackage{longtable}
\usepackage{tabulary}
\usepackage{booktabs,array,multirow}
\usepackage{amsfonts,amsmath,amssymb}
\providecommand\citet{\cite}
\providecommand\citep{\cite}
\providecommand\citealt{\cite}
% You can conditionalize code for latexml or normal latex using this.
\newif\iflatexml\latexmlfalse
\providecommand{\tightlist}{\setlength{\itemsep}{0pt}\setlength{\parskip}{0pt}}%

\AtBeginDocument{\DeclareGraphicsExtensions{.pdf,.PDF,.eps,.EPS,.png,.PNG,.tif,.TIF,.jpg,.JPG,.jpeg,.JPEG}}

\usepackage[utf8]{inputenc}
\usepackage[english]{babel}








\begin{document}

\title{UIUC Plant Physiology Journal Club: 2018-09-05}



\author[1]{Steven Burgess}%
\affil[1]{University of Illinois at Urbana–Champaign}%


\vspace{-1em}



  \date{\today}


\begingroup
\let\center\flushleft
\let\endcenter\endflushleft
\maketitle
\endgroup









\emph{~Iulia Floristeanu, ~\emph{Cindy Chan}~, Steven Burgess
(0000-0003-2353-7794), Charles Pignon, Stephanie Cullum, Isla Causon,
Pietro Hughes}

\textbf{Abstract}

In the preprint ``StomataCounter: a deep learning method applied to
automatic stomatal identification and counting''
(doi:~\url{https://doi.org/10.1101/327494}) Fetter et al. introduced a
reliable and automated stomata counting program that is more efficient
and accurate than human counting and existing algorithms, with a low
false positive rate. The authors report the algorithm can be used for
previous uncharacterised species and has a 94.2\% transfer accuracy when
used on untrained datasets. In addition they provide a publically
available webtool which could be very beneficial for researchers working
on stomata.

\par\null

\textbf{Review}

We really enjoyed the paper and found it to be of high interest as the
method presented could make the work of a lot of people easier. We were
particularly impressed with the precision score of StomataCounter which
compares well with other existing algorithms, especially in terms of the
transfer accuracy. To our knowledge this is a novel approach, as there
are no automated stomata counting technologies available, and it
outperforms existing methods. The DCCN appears to overcomes challenges
of stomata counting and it correctly identified stomata showing minimal
false positives on non-plant and non-stomata covered tissue.

\par\null

The article was well written and easy to follow. We liked the choice of
a Deep Convolutional Neural Network (DCNN) for machine learning and felt
the authors could have highlighted the benefits of this method by
providing a more detailed justification of why it is superior to other
algorithms - as this is what made the paper so interesting to us. The
text might be further improved by being more specific about results in
the abstract and a clearer statement on supplementary data and sampling
used.

\par\null

We were interested to know how the algorithm performs on grass species,
as they have ``dumbbell-shaped'' guard cells and companion cells. It was
unclear to us whether grass stomata have been used among the training
images so we wondered if the accuracy would be the same for this
particular shape of stomata. Including some text in the discussion about
this would be illuminating. In addition for the sake of reproducibility
and to aid readers comprehension it would be useful to provide (1) a
complete list of samples analysed and (2) ideally the whole training
dataset in a public repository such as Zendo or Dryad, as it could
greatly benefit future comparative studies

\par\null

\textbf{Questions we had it would help to clarify}

\begin{itemize}
\tightlist
\item
  The authors state ``many researchers are likely to manually count
  stomata'' - ~is this because other methods are not good enough, user
  interface is too complicated or just people liking more traditional
  things?
\item
  In the methods it is written ``final fully connected layers by
  convolutions''. Does this mean there's no fully-connected layers, only
  convolutional layers? It sounds like both are included in other parts
  of the paper.
\item
  We were confused by the statement ``we argue that the current
  architecture does not transfer well to different scales and images
  should be pre-processed to match the training scale of the network,''
  is that not the reason a fully-convolutional network is used? to avoid
  the inability to manage different input sizes.
\end{itemize}

\textbf{Minor comments}

\begin{itemize}
\tightlist
\item
  Consider making legends of Figure 3 and Figure 4 bigger to improve
  readability
\end{itemize}

\par\null

\begin{itemize}
\tightlist
\item
  Figure 5 is very informative and summarizes the data well
\item
  Figure 6 might be improved by including an inset to focus on the down
  number.
\item
  Figure 7 - scale bars
\item
  There was a strong linear correlation at higher magnification between
  DCNN and human counts is this a standard magnification for this type
  of experiment?
\item
  It would be good to include:
\item
  Quantification of the accuracy of the automatic separation between
  abaxial and adaxial datasets. A failure of separation could
  potentially have contributed to finding stomata on the ``adaxial''
  cuticle.
\item
  Discussion of whether the magnifications (200x and 400x) are suitable
  for this kind of analysis.
\item
  Information about why the 4 sources of images were chosen and
  discussion of whether they provide enough of a range.
\item
  An explanation of why data was which showed \textless{}98\% was
  discarded
\item
  Stats on Gingko and poplar datasets.
\item
  Might consider rephrasing ``the mean number of stomata detected in the
  adaxial, aorta, and breast cancer image sets was 1.5, 1.4, and 2.4,
  respectively, while the ~mean value of the abaxial set was 24.1'' to
  better explain the low frequency: to compare with non-stomata dataset
  vs dataset that has very low number of stomata
\item
  Code availability: It would be helpful if the code and custom scripts
  (such as separation of abaxial and adaxial leaf sides) are made
  available and linked to a repository.
\end{itemize}

\selectlanguage{english}
\FloatBarrier
\end{document}

