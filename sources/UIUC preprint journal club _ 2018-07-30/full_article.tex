\documentclass[10pt]{article}

\usepackage{fullpage}
\usepackage{setspace}
\usepackage{parskip}
\usepackage{titlesec}
\usepackage{placeins}
\usepackage{xcolor}
\usepackage{breakcites}
\usepackage{lineno}





\PassOptionsToPackage{hyphens}{url}
\usepackage[colorlinks = true,
            linkcolor = blue,
            urlcolor  = blue,
            citecolor = blue,
            anchorcolor = blue]{hyperref}
\usepackage{etoolbox}
\makeatletter
\patchcmd\@combinedblfloats{\box\@outputbox}{\unvbox\@outputbox}{}{%
  \errmessage{\noexpand\@combinedblfloats could not be patched}%
}%
\makeatother


\usepackage[round]{natbib}
\let\cite\citep




\renewenvironment{abstract}
  {{\bfseries\noindent{\abstractname}\par\nobreak}\footnotesize}
  {\bigskip}

\renewenvironment{quote}
  {\begin{tabular}{|p{13cm}}}
  {\end{tabular}}

\titlespacing{\section}{0pt}{*3}{*1}
\titlespacing{\subsection}{0pt}{*2}{*0.5}
\titlespacing{\subsubsection}{0pt}{*1.5}{0pt}


\usepackage{authblk}


\usepackage{graphicx}
\usepackage[space]{grffile}
\usepackage{latexsym}
\usepackage{textcomp}
\usepackage{longtable}
\usepackage{tabulary}
\usepackage{booktabs,array,multirow}
\usepackage{amsfonts,amsmath,amssymb}
\providecommand\citet{\cite}
\providecommand\citep{\cite}
\providecommand\citealt{\cite}
% You can conditionalize code for latexml or normal latex using this.
\newif\iflatexml\latexmlfalse
\providecommand{\tightlist}{\setlength{\itemsep}{0pt}\setlength{\parskip}{0pt}}%

\AtBeginDocument{\DeclareGraphicsExtensions{.pdf,.PDF,.eps,.EPS,.png,.PNG,.tif,.TIF,.jpg,.JPG,.jpeg,.JPEG}}

\usepackage[utf8]{inputenc}
\usepackage[english]{babel}








\begin{document}

\title{UIUC preprint journal club : 2018-07-30~~}



\author[1]{Steven Burgess}%
\affil[1]{University of Illinois at Urbana–Champaign}%


\vspace{-1em}



  \date{\today}


\begingroup
\let\center\flushleft
\let\endcenter\endflushleft
\maketitle
\endgroup









\emph{Steven Burgess (0000-0003-2353-7794), Samuel Fernandes, Antony
Digrado, Charles Pignon, Elsa de Becker, Naomi Housego Day, Lusya
Manukyan, Stephanie Cullum, Isla Causon, Iulia Floristeanu, Young Cho,
Freya Way, Judy Savitskya, Robert Collison, Aoife Sweeney, Pietro
Hughes, Cindy Chan}

\section*{Abstract}

{\label{752720}}

This review is compiled from notes taken by the UIUC Plant Physiology
preprint journal club during a one hour session on 2018-07-30. The
review refers to the preprint ``Natural variation in stomata size
contributes to the local adaptation of water-use efficiency in
Arabidopsis thaliana'' by Hannes Dittberner, Arthur Korte, Tabea
Mettler-Altmann, Andreas Weber, Grey
Monroe,\href{http://orcid.org/0000-0002-2942-4750}{}and Juliette de
Meaux ( doi: \url{https://doi.org/10.1101/253021}) published on BioRxiv.

\par\null

\section*{Review}

{\label{304309}}

In the paper ``Natural variation in stomata size contributes to the
local adaptation of Water use efficiency in Arabidopsis thaliana''
Dittberner et al. (2018) investigated the genetic basis of water use
efficiency using genome wide association analysis. Findings included (1)
that there is substantial variation in stomatal size and patterning in
A. thaliana accessions (2) decreased stomata size correlates with
increased water use efficiency (3) two novel QTL affecting WUE
independently of stomatal patterning (4) water use efficiency is a
polygenic trait and (5) natural selection contributed to the
establishment of variation in WUE in accordance with climatic
conditions.

\par\null

The findings are in accordance with previous studies looking at genetic
variation in stomatal size and patterning in A. thaliana (Delgado et al.
2011; Monda et al. 2016), the role of natural selection in WUE of tomato
(Muir et al. 2014), analysis of genomic variation in WUE in A. thaliana
(Easlon et al. 2014; Aliniaeifard and van Meeteren 2014) and the
extensive literature correlating a decrease in stomatal size with
increased water use efficiency in many species.

We were impressed by the significant technical advance presented in the
form of automated stomatal counting and suggest more could be made of
this aspect of the paper. For example the we suggest inclusion of the
term `high-throughput' in the title, and think it would be helpful to
include a picture of how the screening system works in the introduction.

\par\null

Further, we found the conclusion that water use efficiency is a
polygenic trait very interesting. The implications are important for
studies aimed at the improvement of photosynthesis/WUE through
conventional breeding or genetic manipulation. We wonder if there are
other papers which have already discussed the potential polygenic nature
of stomatal traits? Yoo et al. (2011) mentions different genes involved
in WUE and stomatal patterning. This kind of paper could have been
worth-mentioning in the discussion.

\par\null

The identification of two novel QTL, not otherwise known to influence
water use efficiency or to be related to stomatal patterning and
regulation was particularly intriguing, and these two genes provide a
great starting point for further analysis in future publications.

\par\null

\subsection*{Major comments}

{\label{289590}}

We believe the kinship matrix accounts for relatedness, and is usually
called the k model. To account for population structure one could use
principal components, combining these two approaches is known as the Q+K
model. It would be helpful to include a citation for calculating the
kinship matrix as there are multiple methods available. It might also be
worth providing QQplots to show whether including population structure
is necessary or not for this study.

We found the use of a Bonferroni correction in the analysis quite
conservative, there are other options such as Benjamini-Hochberg
adjusted p-values that could lead to declaring other SNPs as
significant. Also, from the GWAS results, there seem to be other SNPs
popping up - although, not significant. It might be worth using an
alternative to p-values when investigating the possible biological
significance of outstanding SNPs on a Manhattan plot - such as selecting
the top 10 hits for investigation.

\par\null

\subsection*{Minor comments:}

{\label{728190}}

\begin{itemize}
\tightlist
\item
  Keywords: Consider revisiting, as many of the keywords are already on
  the title. Additionally, it would be helpful to include terms rather
  than acronyms which are not familiar to non-experts.
\item
  Introduction: Would be good to include examples of some of the
  limitations to automated confocal microscopy approaches to orientate
  readers.
\item
  Line 143: (Atwell et al., 2010). This citation doesn't seem to be
  right. If the author is referring to genomic heritability it could
  cite de los Campos et al. (2015).
\item
  Methods: Would like to see further information on the methodology used
  for carbon isotope discrimination in the main text
\item
  Results: Might consider using the term `genetic-heritability' rather
  than `pseudo-heritability'
\item
  Results: Direct comparison with previously reported dC13 values for
  accessions that have been demonstrated to differing WUE would be
  helpful.
\item
  Figure 4: the text on the figure is likely to be too small to see
  properly
\end{itemize}

\begin{itemize}
\tightlist
\item
  Line 505: ``soil composition\ldots{}'' This is a really interesting
  point about the complexity of environmental effects on plant
  physiology. It might be helpful to provide some explanation for
  readers less familiar with soil content. (e.g. water holding content
  and effect on root structure -
  e.g.\url{https://doi.org/10.1073/pnas.1721749115})
\item
  Line 499-500. Repetition of text on 498-499
\item
  Results: It would be helpful to share where the functional annotation
  for genes come from. This gene appears to be involved in mRNA
  decapping. There is only one publication, which found it is involved
  in PAMP triggered immunity, plants were dwarf in stature. TAIR does
  not annotate a role in cell differentiation (10.15252/embj.201488645)
\item
  Discussion: It is interesting to see a comparison with a related
  species and how the observed pattern is consistent between the two.
\end{itemize}

\begin{itemize}
\tightlist
\item
  Data accessibility: This is excellent. Great to see that all data and
  code will be made available and deposited in a permanent repository!
\end{itemize}

\subsection*{References}

{\label{424989}}

Aliniaeifard and van Meeteren (2014) J Exp Bot. 65(22): 6529--6542.

de los Campos et al. (2015) PLoS Genet 11(5): e1005048.
doi:10.1371/journal.pgen.1005048

Delgado et al. (2011) Ann Bot.; 107(8): 1247--1258

Easlon et al. (2014) Photosynth Res 119: 119.
https://doi.org/10.1007/s11120-013-9891-5

Monda et al. (2016) Plant Physiol, 170: 1435--1444

Muir et al. (2016) Genetics. 2014 Dec; 198(4): 1629--1643.

\textbf{}

\textbf{}

\textbf{}

\selectlanguage{english}
\FloatBarrier
\end{document}

