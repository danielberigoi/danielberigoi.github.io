\documentclass[10pt]{article}

\usepackage{fullpage}
\usepackage{setspace}
\usepackage{parskip}
\usepackage{titlesec}
\usepackage{placeins}
\usepackage{xcolor}
\usepackage{breakcites}
\usepackage{lineno}





\PassOptionsToPackage{hyphens}{url}
\usepackage[colorlinks = true,
            linkcolor = blue,
            urlcolor  = blue,
            citecolor = blue,
            anchorcolor = blue]{hyperref}
\usepackage{etoolbox}
\makeatletter
\patchcmd\@combinedblfloats{\box\@outputbox}{\unvbox\@outputbox}{}{%
  \errmessage{\noexpand\@combinedblfloats could not be patched}%
}%
\makeatother


\usepackage[round]{natbib}
\let\cite\citep




\renewenvironment{abstract}
  {{\bfseries\noindent{\abstractname}\par\nobreak}\footnotesize}
  {\bigskip}

\renewenvironment{quote}
  {\begin{tabular}{|p{13cm}}}
  {\end{tabular}}

\titlespacing{\section}{0pt}{*3}{*1}
\titlespacing{\subsection}{0pt}{*2}{*0.5}
\titlespacing{\subsubsection}{0pt}{*1.5}{0pt}


\usepackage{authblk}


\usepackage{graphicx}
\usepackage[space]{grffile}
\usepackage{latexsym}
\usepackage{textcomp}
\usepackage{longtable}
\usepackage{tabulary}
\usepackage{booktabs,array,multirow}
\usepackage{amsfonts,amsmath,amssymb}
\providecommand\citet{\cite}
\providecommand\citep{\cite}
\providecommand\citealt{\cite}
% You can conditionalize code for latexml or normal latex using this.
\newif\iflatexml\latexmlfalse
\providecommand{\tightlist}{\setlength{\itemsep}{0pt}\setlength{\parskip}{0pt}}%

\AtBeginDocument{\DeclareGraphicsExtensions{.pdf,.PDF,.eps,.EPS,.png,.PNG,.tif,.TIF,.jpg,.JPG,.jpeg,.JPEG}}

\usepackage[utf8]{inputenc}
\usepackage[english]{babel}








\begin{document}

\title{PREreview of OA Week Live-streamed Neuroscience preprint JC}



\author[1]{Alexander Morley}%
\author[2]{cuili }%
\author[3]{Luciana Gallo}%
\author[4]{Alison DePew}%
\author[5]{Mike Aimino}%
\author[4]{Samantha Hindle}%
\author[5]{Tim Mosca}%
\affil[1]{University of Oxford}%
\affil[2]{University of Rochester}%
\affil[3]{University of Buenos Aires}%
\affil[4]{PREreview}%
\affil[5]{Thomas Jefferson University}%


\vspace{-1em}



  \date{\today}


\begingroup
\let\center\flushleft
\let\endcenter\endflushleft
\maketitle
\endgroup





\selectlanguage{english}
\begin{abstract}
This is a review of the bioRxiv preprint~``Sex Differences in
Aggression: Differential Roles of 5-HT2, Neuropeptide F and Tachykinin''
by~Andrew N~Bubak,~Michael J~Watt,~Kenneth J~Renner,~Abigail
A~Luman,~Jamie D~Costabile,~Erin
J~Sanders,~\href{http://orcid.org/0000-0002-8563-9182}{~}Jaime L~Grace,
and John~Swallow. DOI:~\url{https://doi.org/10.1101/407478}~This review
was compiled from discussion points raised during a PREreview
live-streamed preprint journal club as part of Open Access Week, 2018.
The event details can be
found~\href{https://prereview.org/users/153686/articles/325778-prereview-plos-open-access-week-preprint-journal-club-information}{here}~and
the collaborative Etherpad showing all the journal club notes can be
found
\href{https://etherpad.net/p/NeuroscienceLiveStreamedPREJC}{here}.~

\par\null

In addition to those named as authors above, the participants who wished
to be acknowledged for their contributions to this review are as
follows:~Dariusz K. Murakowski,~Tim Koder,~and Daniela Saderi.%
\end{abstract}%




\par\null

\section*{Overview and take-home
messages:}

{\label{392870}}

The study looked at the differential changes in behaviour following
manipulations of both gene expression and the local environment of the
stalk-eyed fly, in particular socialization, following a 7-day period of
isolation. The genes of interest are all related to Serotonin signalling
- a neurotransmitter linked to aggression in both invertebrates and
vertebrates (perhaps in opposite ways). The authors found that 5-HT2
receptors played an inhibitory role in regulating the~\emph{initiation}
of aggressive behaviours using siRNA knockdown experiments. In contrast,
higher~\emph{escalation} of aggressive behaviours was observed following
application of 5-HT, which the authors speculate was linked to increase
Tk expression. Notably both decreased 5-HT2 receptor expression and
increased Tk expression were observed in socially isolated males, who
were also more aggressive in both measures suggesting that these two
parallel pathways might confer a survival advantage under these
environmental conditions. The authors did not observe these effects in
female stalk-eyed flies, but it is unclear whether this results from
different aggression pathways, or is a reflection of the specific
behavioural assays used to assess aggression in this study.

\par\null

\section*{Positive remarks:}

{\label{260841}}

The journal club participants were surprised that such a ubiquitous
neurotransmitter can have such disparate effects across organisms and
sexes, and led to the question of how we are able to create such
different behaviours using a single molecule. Participants were
particularly intrigued by the finding that~different sub-types of 5-HT
receptors exert opposing effects according to the context. This led to
questions about differences in sub-type affinity and any potential
impact on sex differences of expression. Furthermore, it was an
interesting finding that there were sex-specific effects on aggression
after isolation; in particular that size differences affect the
aggressive behaviour of the male fly vs. a lack of aggressive behaviour
regardless of size or rearing for females. We know so little about the
associated receptors: where and how much they are expressed. There is
still so much to investigate. This is one great aspect of this paper.~

\par\null

\section*{Constructive feedback:}

{\label{731764}}

\subsection*{Visualisation /
Presentation}

{\label{233033}}

\begin{itemize}
\tightlist
\item
  How many people know how cool the stalk-eyed fly is? A simple photo
  and high-speed video recordings would be a really useful addition to
  the paper. Furthermore, it would be very nice to see a video that
  shows how the behaviour changes after the manipulations.
\item
  It was difficult to visualize the behavioural types (initiation,
  high-intensity, etc.). Therefore, a picture of each type would have
  been helpful for the reader, together with additional behavioural
  descriptions. As above, a video/picture would say a thousand words.
\end{itemize}

\par\null

\subsection*{Background / Structure}

{\label{317172}}

\begin{itemize}
\tightlist
\item
  The introduction was difficult to follow at the beginning. It would be
  useful to reveal what is known from the field and discuss how
  that~related to this organism.
\item
  It would be really useful to add to the introduction some of the ideas
  that were in the discussion, for example, the aims and justifications
  for using this species.~
\item
  Every piece of data was treated with the same importance. It would be
  useful to break the data apart a little to emphasize the important
  results, and place other less important results in an
  Appendix/Supplemental Figure/Table. It is difficult~ to parse out the
  importance at present, so this would help increase the clarity of the
  paper.~
\item
  Although Table 2 was useful to help summarize the findings of this
  manuscript, a model diagram/visual would be helpful, particularly to
  bring these findings into context with what is known in other
  invertebrates and vertebrates. This could be placed at the end of the
  manuscript, but referred to in the introduction also (re: what is
  already known in the literature).
\item
  In line with the above, it would be helpful if each result can be
  contrasted with a species that is already known in order to give
  context. Furthermore, when different organisms were mentioned, the
  connection and comparison between them was not always clear.~ The
  referential data changed species from one section to the next, causing
  some confusion. Therefore, it may be helpful to focus on one or two
  species for clarity.~
\end{itemize}

\subsection*{Technical Questions /
Controls}

{\label{980637}}

\begin{itemize}
\tightlist
\item
  It seems possible that the test for aggression is only ethnologically
  relevant for males. Is there a preferred test of aggression for sex
  difference, or are there other conditions, for example
  maternality-related aggression tests, that would be able to pull out
  differences in the effects of manipulations for females? It would be
  useful to be able to generalize across aggressive behaviours,
  otherwise it is important to state in the discussion that the results
  apply to this prototype of behaviour specifically.~In other words, the
  authors did not observe these aggressive effects in female stalk-eyed
  flies, but it is unclear whether this results from different
  aggression pathways, or is a reflection of the specific behavioural
  assays used to assess aggression in this study. This should be
  commented on in the discussion.~
\item
  For the forced fight paradigm, specifically the siRNA and 5-HTP
  experiments, were the opponents socially reared? Furthermore, were the
  opponents~ familiar with each other? In some species, aggression
  differs toward familiar and non-familiar conspecifics, so this could
  potentially change the outcome of the fight.
\item
  Have the authors considered, and/or tested, whether a change in the
  amount of movement may have caused some of these differences in
  aggression? Having the video footage might help us understand this
  more. It is possible that some lower level behaviours are being missed
  that would be captured by video analyses.
\item
  Distinction between aggressive bout initiation vs the intensity of the
  individual experience was a very interesting element and the
  manuscript would have benefited from emphasizing this more. This is a
  curious finding, and we would have liked to know more. The high speed
  videoing may also help with this and could allow the incorporation of
  machine learning/AI to help distinguish between these two types of
  behavioural responses. An example of this type of software would be
  `deeplabcut' (Mathis et al. Nat Neurosci, 2018).
\item
  If~the 5-HT receptors are~ similar between~\emph{Drosophila} and the
  stalk-eyed fly, could reagents (particularly immunohistochemistry) be
  used in the stalk-eyed fly to look at receptor sub-type
  localization/co-localization, and protein levels? If so, this might
  help the authors compare their results with the \emph{Drosophila}
  literature.
\item
  More specific antagonists/agonists for these receptors are needed as
  the pharmacology is limited.
\end{itemize}

\par\null

\subsection*{Questions for the authors:}

{\label{996460}}

\begin{itemize}
\tightlist
\item
  The manuscript could have benefited from more details on where we
  stand before and where we are after this preprint. There is a lot of
  data that can be seen as conflicting across different model system and
  it would be useful to discuss this more. Furthermore, fundamental
  questions like ``What is serotonin doing?'' and ``How do you reconcile
  what happens in invertebrates with other findings in vertebrates?''
  would be valuable additions to this manuscript. Can we really start to
  parse out some of these elements on how we can have multiple different
  effects with the same neurotransmitter? It would be really interested
  to know what the different evolutionary roles of aggression are in
  these model and non-model organisms. Might this underlie some of the
  differences?
\item
  By looking at different receptor sub-types, were the authors trying to
  address behavioural differences between specific populations of
  neurons, or were they trying to suggest that serotonin is having a
  global influence on aggression? If so, this would suggest that
  serotonin acts in specific neurons to modulate different aspects of
  aggression depending on which population of neurons it is acting in.
  Maybe this is something the authors could comment on, or build upon
  for future experiments.
\item
  Some of the figures show changes in the levels of neuropeptides for
  the females either after isolation or after treatments, but these
  changes are not significant. It would be interesting if the authors
  could expand upon possible minor effects these changes may cause and
  if they might be affecting other behaviours, even if the changes are
  not drastic enough to be significant.
\item
  What is the affinity of 5-HT to the different sub-types of receptors?
  If this fact is known, perhaps the authors can include it in the
  discussion.
\item
  How does increased aggression elevate 5-HT2 in the stalked-eyed fly?
\item
  Could you make a fake opponent that a fly might engage or retreat
  from, even if there is a large difference in size? The authors might
  be able to pull apart different responses by doing this.
  Alternatively, the authors could place the flies up against a mirror,
  and see if they fight themselves.
\item
  We wondered whether aggression behaviours would happen across sexes
  when 5-HT receptors are manipulated, or whether the fly would show
  increased mating behaviours with the other sex due to their increased
  aggressive behaviours. The authors may wish to comment on this in the
  discussion.
\item
  Is there a difference in 5-HT2 (or other) expression between males and
  females following 5-HT2 siRNA? {[}see Fig. 4 and 5{]}.
\item
  The manuscript would benefit from showing how many flies were tested
  at each stage, how many were lost from the data due to behavioural
  abnormality, injection failure, death etc? This would increase
  transparency about such aspects and lead to greater reproducibility
  and therefore credibility (note: this is a general point for
  scientific practices rather than a specific point about this
  manuscript).
\item
  Is it possible to elucidate a causal role for Tk and/or NPF by
  injecting/expressing them directly?
\end{itemize}

\subsection*{}

{\label{476186}}

\section*{Minor comments:}

{\label{323425}}

\begin{itemize}
\tightlist
\item
  \textbf{Title:} the authors may consider modifying the title to
  emphasize the key questions that are being addressed by this work. We
  feel this would increase the clarity and impact of this work.
\item
  \textbf{Introduction:} it might be useful to reduce its length by
  grouping the information for the reasons to use the stalk-eyed fly. In
  lines 107-126, the authors demonstrate that the stalk-eyed fly can be
  a useful modeI to detect the conflict (i.e, initiation, escalation and
  termination). In lines 151-171, the authors demonstrate that the
  stalk-eyed fly can be an ideal model for examining inset aggression in
  a sex-dependent manner. It might be beneficial to place the
  information about NPY and sex difference in aggression (lines 127-171)
  in the discussion instead.~
\item
  \textbf{Introduction:} More links to other animals and humans would be
  useful. In lines 132-135: there is a link with mice; though the
  statement may be the wrong way around because reference 2 covers flies
  not mice, and reference 22, mice but not flies.
\item
  \textbf{Line 242:} First use of NPFr should say ``NPF receptor''. If
  abbreviated, NPFr should be consistent throughout the text (see Lines
  333, 334 later in the text).
\item
  \textbf{Materials and Methods, Line 253:} this paragraph appeared to
  be more relevant for the results section than for the methods
  section.~
\item
  \textbf{Line 262:} the reference to Fig. 1 is apparently incorrect.
\item
  \textbf{Line 277:} it would be beneficial to include more details
  about the siRNA procedure. Ideally the authors should make the
  sequence public and also comment on how many siRNAs were used to
  investigate this question. A table or list of all siRNAs used in this
  study could be added to the text.
\item
  \textbf{Line 279:} more information on the cDNA would be useful.
\item
  It would be useful to just show the absolute number of initiations in
  each case (rather than subtracting the number in the vehicle
  condition). The data are very convincing so it would be great to see
  all of it!
\item
  \textbf{Fig. 1:} Simply using a color panel to illustrate the groups
  as females and males would be sufficient, rather than repetitively
  using a caption. Same suggestion for Fig. 3.
\item
  \textbf{Fig. 1-7:} When there are less than 50 observations, it would
  be beneficial to see the individual points for each of the bar charts,
  similar to what is shown in the paired line plots.
\item
  \textbf{Fig. 5D:} This figure is not described nor mentioned in the
  text (lines 411-416).
\item
  \textbf{Fig. 8:} The ``inhibitory'' arrow is unclear. In particular,
  does it mean that 5-HT2 activation inhibits NPF signaling, or does it
  mean that a *reduction* in 5-HT2 activation inhibits NPF signaling?
\item
  \textbf{Discussion:} This research was very insect-centric. More
  information about the relationship of the results to mammals, and the
  relationship to human health etc., would be appreciated.
\item
  \textbf{Discussion:}~The manuscript would benefit from the addition of
  more information about the 5-HT1A and 5-HT2 receptor in other
  organisms during aggression.
\item
  \textbf{Discussion, Line 521:} Why does the isolation protocol
  for~\emph{Drosophila~}differ from the one used in this work? The
  reasons for not using the same protocol could be stated.
\end{itemize}

\par\null

\subsection*{Survey results:~}

{\label{292154}}

As part of the live-streamed journal club, we conducted a short survey
to allow all participants to contribute their overall impressions of the
preprint. Below are the results of each question.

\par\null

\textbf{Question \#1:}

\textbf{In your opinion, how interesting is the research?}

\begin{itemize}
\tightlist
\item
  \textbf{} Not interesting
\item
  Quite interesting
\item
  Very interesting
\end{itemize}

\par\null\selectlanguage{english}
\begin{figure}[h!]
\begin{center}
\includegraphics[width=0.70\columnwidth]{figures/Question1-NeuroLivePRJC-OAWeek18/Question1-NeuroLivePRJC-OAWeek18}
\caption{{Survey question \#1 (n=12 participants)
{\label{789006}}%
}}
\end{center}
\end{figure}

\textbf{Question \#2:}

\textbf{Which statement best describes the methods?}

\begin{itemize}
\tightlist
\item
  Methods are incomplete or poorly written. There are insufficient
  replicates, and/or there are no (or incorrect) statistical analyses
\item
  Methods are well written with enough detail for the overall procedure
  to be understood.~Statistical significance is calculated
\item
  Methods are complete and written with enough detail to be replicated,
  including reagents and strains used (if appropriate). Statistical
  significance is calculated using appropriate tests
\end{itemize}

\par\null\par\null\par\null\selectlanguage{english}
\begin{figure}[h!]
\begin{center}
\includegraphics[width=0.70\columnwidth]{figures/Question2-NeuroLivePRJC-OAWeek18/Question2-NeuroLivePRJC-OAWeek18}
\caption{{Survey question \#2 (n= 11 participants)
{\label{979776}}%
}}
\end{center}
\end{figure}

\textbf{Question \#3:}

\textbf{Were the figures understandable without having to read the whole
manuscript text?}

\begin{itemize}
\tightlist
\item
  No, the figures/tables and/or legends were difficult to understand
  without reading the whole manuscript
\item
  Some of the figures/tables and/or legends were understandable without
  reading the whole manuscript
\item
  Yes, all of the figures/tables and/or legends were understandable
  without reading the whole manuscript
\end{itemize}

\par\null\selectlanguage{english}
\begin{figure}[h!]
\begin{center}
\includegraphics[width=0.70\columnwidth]{figures/Question3-NeuroLivePRJC-OAWeek18/Question3-NeuroLivePRJC-OAWeek18}
\caption{{Survey question \#3 (n= 10 participants)
{\label{547306}}%
}}
\end{center}
\end{figure}

\textbf{Question \#4:}

\textbf{Do the results support the conclusions?}

\begin{itemize}
\tightlist
\item
  No, the results do not support any of the conclusions
\item
  The results support some of the conclusions
\item
  Yes, the results support all of the conclusions
\end{itemize}

\par\null\selectlanguage{english}
\begin{figure}[h!]
\begin{center}
\includegraphics[width=0.70\columnwidth]{figures/Question4-NeuroLivePRJC-OAWeek18/Question4-NeuroLivePRJC-OAWeek18}
\caption{{Survey question \#4 (n=11 participants)
{\label{954383}}%
}}
\end{center}
\end{figure}

\selectlanguage{english}
\FloatBarrier
\end{document}

