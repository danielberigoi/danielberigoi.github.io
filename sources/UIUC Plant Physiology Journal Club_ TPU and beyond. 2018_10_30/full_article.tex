\documentclass[10pt]{article}

\usepackage{fullpage}
\usepackage{setspace}
\usepackage{parskip}
\usepackage{titlesec}
\usepackage{placeins}
\usepackage{xcolor}
\usepackage{breakcites}
\usepackage{lineno}





\PassOptionsToPackage{hyphens}{url}
\usepackage[colorlinks = true,
            linkcolor = blue,
            urlcolor  = blue,
            citecolor = blue,
            anchorcolor = blue]{hyperref}
\usepackage{etoolbox}
\makeatletter
\patchcmd\@combinedblfloats{\box\@outputbox}{\unvbox\@outputbox}{}{%
  \errmessage{\noexpand\@combinedblfloats could not be patched}%
}%
\makeatother


\usepackage[round]{natbib}
\let\cite\citep




\renewenvironment{abstract}
  {{\bfseries\noindent{\abstractname}\par\nobreak}\footnotesize}
  {\bigskip}

\renewenvironment{quote}
  {\begin{tabular}{|p{13cm}}}
  {\end{tabular}}

\titlespacing{\section}{0pt}{*3}{*1}
\titlespacing{\subsection}{0pt}{*2}{*0.5}
\titlespacing{\subsubsection}{0pt}{*1.5}{0pt}


\usepackage{authblk}


\usepackage{graphicx}
\usepackage[space]{grffile}
\usepackage{latexsym}
\usepackage{textcomp}
\usepackage{longtable}
\usepackage{tabulary}
\usepackage{booktabs,array,multirow}
\usepackage{amsfonts,amsmath,amssymb}
\providecommand\citet{\cite}
\providecommand\citep{\cite}
\providecommand\citealt{\cite}
% You can conditionalize code for latexml or normal latex using this.
\newif\iflatexml\latexmlfalse
\providecommand{\tightlist}{\setlength{\itemsep}{0pt}\setlength{\parskip}{0pt}}%

\AtBeginDocument{\DeclareGraphicsExtensions{.pdf,.PDF,.eps,.EPS,.png,.PNG,.tif,.TIF,.jpg,.JPG,.jpeg,.JPEG}}

\usepackage[utf8]{inputenc}
\usepackage[english]{babel}








\begin{document}

\title{UIUC Plant Physiology Journal Club: TPU and beyond. 2018/10/30}



\author[1]{Steven Burgess}%
\affil[1]{University of Illinois at Urbana–Champaign}%


\vspace{-1em}



  \date{\today}


\begingroup
\let\center\flushleft
\let\endcenter\endflushleft
\maketitle
\endgroup









The UIUC Plant physiology journal club chose to review the preprint
``Triose phosphate utilization and beyond: from photosynthesis to
end-product synthesis'' (\url{http://dx.doi.org/10.1101/434928}), by
McClain and Sharkey, about the role of phosphate levels in relation to
CO2 assimilation during photosynthesis. The paper explains that when the
rate of photophosphorylation which produces ATP, exceeds that of starch
or sucrose synthesis which consumes ATP, phosphate levels drop causing
photosynthetic electron transport rates to slow. This process is defined
as Triose phosphate limitation (TPU).

\textbf{}

We found the paper provided a very thorough definition of TPU and
thought it was a really good introduction to the topic. In particular,
the explanation about the way TPU limitation can be identified by no
change in photosynthesis and decline in PhiPSII at high CO2 was very
useful.~The paper raises a number of points we were interested to
discuss further including why TPU limitation is only slightly more than
photosynthetic rate? Are some plants are more subjected to TPU
limitation than others? Why is TPU limitation is slightly more than
photosynthetic rate (one might predict that natural selection would
select for utilization of all their photosynthate)? Why C4 plants do not
experience any TPU limitation? Is it more common to see TPU limitation
in Li6800s?

\par\null

The question of the potential impact of a rising atmospheric CO2
concentrations on TPU is a fascinating topic.~ We thought it would be
beneficial for readers to include reference to Experimental free air CO2
enrichment (FACE) experiments which have sought to address this issue,
and some degree of guidance about what types of increases in atmospheric
CO2 would be required to potentially see an effect.~

\par\null

In addition it would be useful to mention that many of the TPU studies
have been performed in pot bound plants, which experience an altered
shoot/root ratio creating an artificial sink limitation, and a bit more
discussion about how often TPU limitation might occur under field
conditions (Arp 1991). Finally it might be good to include a bit more
discussion about how the xanthophyll cycle kicks in to protect the plant
under stress and that decreased PhiPSII could be an example of
non-photochemical quenching.

\par\null

In summary, we thoroughly enjoyed the paper, we learnt a lot and look
forward to seeing it published.

\subsubsection*{}

\subsection*{Minor points}

{\label{493224}}

\begin{itemize}
\tightlist
\item
  In equation 5 it is not entirely clear what the phi symbol is? ~
\item
  The symbol R is used to represent three different variables, could
  maybe consider changing this to prevent confusion.
\item
  Many color combinations used in the figures would be unsuitable for
  those with color blindness, and suggest using color oracle
  (\url{https://colororacle.org/}) to help.
\item
  Fig 2: Clarification on what `a little' and `a lot' means would be
  helpful
\item
  Fig 2: We found it a bit confusing, many members were unclear about
  what it was demonstrating and suggest considering reformulating the
  description as we believe it could be useful as a teaching aid.
\item
  Fig 3: Many members really liked this figure and thought it provided a
  clear image to explain the various limitations a plant experiences
  during an A/Ci curve. There was some discussion about where the
  boundary lines were drawn, this might be addressed by having zones of
  colour fade into each other rather than represented as sharp
  boundaries. It was also suggested that some kind of upper bound to J
  limitation of ETR might be included (possibly fading out to white?)
\end{itemize}

\subsection*{References}

{\label{362547}}

Arp (1991) \url{https://doi.org/10.1111/j.1365-3040.1991.tb01450.x}

Rodgers et al. (2004)
\url{https://doi.org/10.1111/j.1365-3040.2004.01163.x}

\selectlanguage{english}
\FloatBarrier
\end{document}

