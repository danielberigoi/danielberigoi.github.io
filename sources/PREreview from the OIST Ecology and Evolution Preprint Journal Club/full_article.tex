\documentclass[10pt]{article}

\usepackage{fullpage}
\usepackage{setspace}
\usepackage{parskip}
\usepackage{titlesec}
\usepackage{placeins}
\usepackage{xcolor}
\usepackage{breakcites}
\usepackage{lineno}





\PassOptionsToPackage{hyphens}{url}
\usepackage[colorlinks = true,
            linkcolor = blue,
            urlcolor  = blue,
            citecolor = blue,
            anchorcolor = blue]{hyperref}
\usepackage{etoolbox}
\makeatletter
\patchcmd\@combinedblfloats{\box\@outputbox}{\unvbox\@outputbox}{}{%
  \errmessage{\noexpand\@combinedblfloats could not be patched}%
}%
\makeatother


\usepackage[round]{natbib}
\let\cite\citep




\renewenvironment{abstract}
  {{\bfseries\noindent{\abstractname}\par\nobreak}\footnotesize}
  {\bigskip}

\renewenvironment{quote}
  {\begin{tabular}{|p{13cm}}}
  {\end{tabular}}

\titlespacing{\section}{0pt}{*3}{*1}
\titlespacing{\subsection}{0pt}{*2}{*0.5}
\titlespacing{\subsubsection}{0pt}{*1.5}{0pt}


\usepackage{authblk}


\usepackage{graphicx}
\usepackage[space]{grffile}
\usepackage{latexsym}
\usepackage{textcomp}
\usepackage{longtable}
\usepackage{tabulary}
\usepackage{booktabs,array,multirow}
\usepackage{amsfonts,amsmath,amssymb}
\providecommand\citet{\cite}
\providecommand\citep{\cite}
\providecommand\citealt{\cite}
% You can conditionalize code for latexml or normal latex using this.
\newif\iflatexml\latexmlfalse
\providecommand{\tightlist}{\setlength{\itemsep}{0pt}\setlength{\parskip}{0pt}}%

\AtBeginDocument{\DeclareGraphicsExtensions{.pdf,.PDF,.eps,.EPS,.png,.PNG,.tif,.TIF,.jpg,.JPG,.jpeg,.JPEG}}

\usepackage[utf8]{inputenc}
\usepackage[english]{babel}








\begin{document}

\title{PREreview from the OIST Ecology and Evolution Preprint Journal Club}



\author[1]{maggi brisbin}%
\author[1]{Yuka Suzuki}%
\author[1]{Maki Thomas}%
\author[1]{Sam Ross}%
\author[1]{Julian Katzke}%
\author[1]{Stefano Pascarelli}%
\author[1]{Vienna Kowallik}%
\author[1]{Maeva Angelique Techer}%
\author[1]{Jigyasa Arora}%
\author[1]{Otis Brunner}%
\author[1]{Darko Cotoras}%
\author[1]{Crystal Clitheroe}%
\author[1]{Fabien Benureau}%
\affil[1]{Okinawa Institute of Science and Technology Graduate School}%


\vspace{-1em}



  \date{\today}


\begingroup
\let\center\flushleft
\let\endcenter\endflushleft
\maketitle
\endgroup









\section*{Recent demographic histories and genetic diversity across
pinnipeds are shaped by anthropogenic interactions and mediated by
ecology and
life-history~~}

{\label{463319}}

\subsection*{Martin
Adam~Stoffel,~Emily~Humble,~Karina~Acevedo-Whitehouse,~Barbara
L.~Chilvers,~Bobette~Dickerson,~Fillipo~Galimberti,~Neil~Gemmell,~Simon
D.~Goldsworthy,~Hazel
J.~Nichols,~Oliver~Krueger,~Sandra~Negro,~Amy~Osborne,~Anneke
J.~Paijmans,~Teresa~Pastor,~Bruce
C.~Robertson,~Simona~Sanvito,~Jennifer~Schultz,~Aaron
B.A.~Shafer,~Jochen B.W.~Wolf,~Joseph I.~Hoffman,~ April 12, 2018,
version 1,
bioRxiv~}

{\label{720048}}

\subsubsection*{\texorpdfstring{doi:~\url{https://doi.org/10.1101/293894}}{doi:~https://doi.org/10.1101/293894}}

{\label{236037}}

\section*{}

{\label{463319}}

Firstly, we thank the authors for their work and for posting it as a
preprint on bioRxiv. This work endeavored to evaluate the occurrence and
intensity of population bottlenecks in a large number of pinniped
species that have been differentially affected by human exploitation.
Population bottlenecks can decrease genetic diversity and adversely
affect the ability of a species to adapt to~modern habitat loss and
climate change. Because historical data is sparse and unreliable, the
authors applied population genetics methods to a large, multi-species
dataset to detect and evaluate past population bottlenecks and then
compared the results to life-history traits and current conservation
status for each species. The results indicate that 11 of the species
included in the analysis experienced a population bottleneck and that
land-breeding pinnipeds are more likely to have experienced a bottleneck
than ice-breeding pinnipeds. While there was not an overall relationship
between IUCN status and past bottleneck events, bottleneck events were
detected for 4 of the 6 endangered species included in the study. The
breadth of this study is especially important, as it represents a first
effort to apply these methods across 30 species in a single analysis.
Our overall impression is that this was a large project using an
extensive amount of data from multiple sources, which then had to be
standardized in order to be analyzed in a novel way. This paper
highlights the benefits of open science and open data, as data from
multiple studies was reused and analyzed in a far broader context than
any single study on a single population or species.

\par\null

This manuscript is exceptionally well-written and uses clear language,
making it both easy and enthralling to read. However, there are a few
small mistakes that caused some confusion. Particularly, the caption of
Figure 4 switches the descriptions for Panels A and B. Additionally, the
figures in the main text are numbered 1:4, 6; it appears that Figure 5
may have been moved to the supplemental materials, but the main figure
numbering was not adjusted accordingly.~

\par\null

All of the figures in the manuscript are very attractive and make good
use of consistent coloring. Figure 1 nicely summarizes many of the main
findings of the paper. This figure would be even better if Panel A
utilized a 2-color scale like Panels B and C. Additionally since
P\textsubscript{bot~}and P\textsubscript{neut} are complementary, we
suggest that only P\textsubscript{bot} needs to be included in Panel C,
which will reduce the size of the figure and make it easier to
interpret. Figure 2 is very clear and intuitive and nicely illustrates
the intensity of population bottlenecks for different
species.~Additionally, the pinniped drawings are beautiful and the use
of original artwork in the paper is commendable. We feel that Figure 4,
which displays the expected correlation between global abundance and
allelic richness does not necessarily need to be included in the main
text. Conversely, we feel that Supplementary Table 1, which contains the
sample size, number of microsatellite loci, and citation for each
species' dataset, is important for readers to have available in the main
text.~Overall, the authors did an outstanding job applying population
genetics techniques appropriately. In particular, the authors made very
good use of the program STRUCTURE. This program was used to detect
population substructure and if detected, the largest genetic cluster was
selected for inclusion in ensuing analyses. This important step prevents
false detection of bottlenecks, which can be a common mistake. It is
also appreciated that the authors chose to examine allelic richness
rather than allelic frequency. We were left with a few lingering
questions about the methods, however. First, we are curious if the
authors acquired raw electropherograms or pre-interpreted genotypes for
the published datasets used and if there were any measures taken to
control for observer bias in interpreting microsatellite genotypes, such
as preparing and running samples from other labs and assessing whether
similar conclusions were reached. We were also curious about the
justification for grouping IUCN categories into ``concern'' ('near
threatened,' `vulnerable,' and `endangered') and ``least concern''
('least concern'), especially since the~\emph{IUCN Red List Categories
and Criteria} groups `near threatened' with `least concern' and
explicitly distinguishes `vulnerable,' `endangered,' and `critically
endangered' as the ``threatened'' categories. We wonder how the results
presented in Figure 6 would be affected by moving the species designated
as `near threatened' out of the ``concern'' category.

\par\null

Lastly, we would have appreciated a more extensive discussion. For
example, the authors describe ice-breeding species as experiencing less
historical exploitation than the land-breeding species, but ice-breeding
species are likely more susceptible to negative impacts from climate
change in the recent past and into the future. The implications of the
findings of the study combined with change in anthropogenic disturbance
patterns and continuing disturbance~ in these habitats into the future
could be addressed in the discussion. We are also very interested in
what the authors perceive as weaknesses in the approaches used, if there
may be alternative interpretations of the results, and especially what
future studies they would suggest based on their results and
conclusions. Again, it was a great pleasure reading this impressive
work. We hope that our comments are useful to the authors and we look
forward to reading the final version when it is published.~

\par\null

Thank you!

\par\null

The OIST Ecology and Evolution Preprint Journal Club~

\par\null\par\null

\selectlanguage{english}
\FloatBarrier
\end{document}

