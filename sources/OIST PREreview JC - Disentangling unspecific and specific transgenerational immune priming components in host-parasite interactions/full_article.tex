\documentclass[10pt]{article}

\usepackage{fullpage}
\usepackage{setspace}
\usepackage{parskip}
\usepackage{titlesec}
\usepackage{placeins}
\usepackage{xcolor}
\usepackage{breakcites}
\usepackage{lineno}





\PassOptionsToPackage{hyphens}{url}
\usepackage[colorlinks = true,
            linkcolor = blue,
            urlcolor  = blue,
            citecolor = blue,
            anchorcolor = blue]{hyperref}
\usepackage{etoolbox}
\makeatletter
\patchcmd\@combinedblfloats{\box\@outputbox}{\unvbox\@outputbox}{}{%
  \errmessage{\noexpand\@combinedblfloats could not be patched}%
}%
\makeatother


\usepackage[round]{natbib}
\let\cite\citep




\renewenvironment{abstract}
  {{\bfseries\noindent{\abstractname}\par\nobreak}\footnotesize}
  {\bigskip}

\renewenvironment{quote}
  {\begin{tabular}{|p{13cm}}}
  {\end{tabular}}

\titlespacing{\section}{0pt}{*3}{*1}
\titlespacing{\subsection}{0pt}{*2}{*0.5}
\titlespacing{\subsubsection}{0pt}{*1.5}{0pt}


\usepackage{authblk}


\usepackage{graphicx}
\usepackage[space]{grffile}
\usepackage{latexsym}
\usepackage{textcomp}
\usepackage{longtable}
\usepackage{tabulary}
\usepackage{booktabs,array,multirow}
\usepackage{amsfonts,amsmath,amssymb}
\providecommand\citet{\cite}
\providecommand\citep{\cite}
\providecommand\citealt{\cite}
% You can conditionalize code for latexml or normal latex using this.
\newif\iflatexml\latexmlfalse
\providecommand{\tightlist}{\setlength{\itemsep}{0pt}\setlength{\parskip}{0pt}}%

\AtBeginDocument{\DeclareGraphicsExtensions{.pdf,.PDF,.eps,.EPS,.png,.PNG,.tif,.TIF,.jpg,.JPG,.jpeg,.JPEG}}

\usepackage[utf8]{inputenc}
\usepackage[english]{babel}








\begin{document}

\title{OIST PREreview JC - ``Disentangling unspecific and specific
transgenerational immune priming components in host-parasite
interactions''}



\author[1]{Maggi Brisbin}%
\author[1]{Yuka Suzuki}%
\author[1]{Julian K}%
\author[1]{Jigyasa Arora}%
\affil[1]{Okinawa Institute of Science and Technology Graduate School}%


\vspace{-1em}



  \date{\today}


\begingroup
\let\center\flushleft
\let\endcenter\endflushleft
\maketitle
\endgroup









\section*{Disentangling unspecific and specific transgenerational immune
priming components in host-parasite
interactions}

{\label{272979}}

Frida Ben-Ami, Christian Orlic, Roland R. Regoes~

doi:~\url{https://doi.org/10.1101/429498}

bioRxiv, 9/27/2018

\par\null

\textbf{Overview and take-home messages:}

In this study, the authors tackle the topic of transgenerational immune
priming in invertebrates. The authors designed a large experiment taking
advantage of clonal~\emph{Daphnia} to test whether infecting parental
generations with different parasite strains improves the offspring's
resistance to that parasite overall and if yes, if they resist that
specific strain more effectively than other strains. This experiment
essentially tests the specificity of immune priming at a very fine
``strain'' scale. The results did not support parental infection strain
differentially affecting offspring resistance to different strains,
suggesting that immune priming is not specific to the strain level in
this system. However, a mathematical model the authors developed for
that study fits the data exceptionally well, which means this model
could potentially be used in a predictive manner for this or similar
systems. Additionally, the unexpected result that one strain actually
facilitates specific infection in the offspring is surprising and opens
the door to additional inquiry and future experimentation. Overall this
study is very interesting and well-presented, but there are a few
concerns that could be addressed and improved in the next version of the
manuscript.

\textbf{Positive feedback:}

\begin{itemize}
\tightlist
\item
  The~\emph{Daphnia-Pasteuria} system is an interesting and appropriate
  system to study transgenerational effects. The fact that Daphnia is
  parthenogenic and produces clonal offspring is an amazing advantage
  that the authors utilize well in this study.~
\item
  Transmitting acquired traits, like immune memory, from parents to
  offspring is an exciting prospect. However, in this direct,
  single-generation transmission, there are many confounding factors.
  Parental infection could affect offspring in myriad ways, such as
  decreased offspring size to name only one. This is especially true in
  this system where the parasite directly influences host reproduction.
  It would be extremely interesting to see the authors extend this work
  into the F2 and F3 generations and determine if immune priming
  persists past the generation directly derived from the infected
  parent.~
\item
  The model applied in this study fits the data very well and is not the
  type of model typically considered by microbial ecologists. Other
  species interactions and specificity studies in different systems
  could benefit greatly by exploring the use of frailty mixed models as
  were adapted for this paper.~~
\end{itemize}

\textbf{Major concerns:}

\begin{itemize}
\tightlist
\item
  The experimental design is very impressive, considering the large
  number of comparisons included in the experiment. A major concern for
  us was the seeming lack of biological replicates, or at least an
  obvious description of them. The methods section indicates that each
  infection experiment was carried out in a 100 mL jar, but it seems
  like there was one jar per treatment combination (it is possible we
  are missing something in our reading). We feel it is important to
  clearly state how many jars were prepared per treatment and if it was
  only one jar per treatment to discuss why this was the case. We
  understand that this is large experiment and it was likely very
  time-consuming to maintain all of the treatments, which could have
  made replication simply not feasible. However, biological replication
  would greatly increase the reliability of the results and it is
  important to recognize and discuss this. This comes especially into
  play when the authors discuss the variability in results between the
  current experiment and a previously published~ experiment. If there is
  biological replication in both experiments, readers can better
  evaluate whether the observed variation was within a normal range of
  variation for this system. Likewise, error bars in Figure 2 would make
  the results easier to evaluate. If we have interpreted the text
  properly and biological replication was not included, the authors
  could consider explaining why this was the case and explicitly how
  they account for it in the results.
\item
  In the introduction, methods, and results section the authors refer to
  parasite ``strains,'' but switch to parasite ``isolates'' in the
  discussion where it becomes clear that this distinction is rather
  important to interpreting the results of the study. When they are
  referred to as strains, we assumed that each strain was a genetically
  distinct and genetically homogenous culture.~ In the discussion, there
  is a more comprehensive description of the ``strains'' and they were
  more accurately described as ``isolates.'' Although the isolates are
  more representative of natural infections, their very nature makes the
  experiments very difficult to replicate. It would be helpful to use
  more controlled isolates (known strain composition) and do 16S
  monitoring to determine if strain or isolate composition itself
  evolves and changes over time .
\end{itemize}

\textbf{Minor concerns:}

\begin{itemize}
\tightlist
\item
  Figure 5 is a very clear and informative figure. As readers, we can
  easily observe that the ID50 decreases two-fold in primed offspring
  compared to the control. In addition, the anomalous increase in P5
  infection is also straightforward. It is more difficult to see these
  trends in Figure 2. The authors might consider including ID50 in this
  plot as well or to remove it completely.
\item
  Although it is very clear that the ID50 changes between treatments, it
  is difficult for us to put the change in context without more
  discussion. Is it a big change? How does the change compare to studies
  in other model organisms?
\item
  We recognize that the model used in this study is a central strong
  point of the work. However, since we were not previously familiar with
  frailty models, it would be helpful to have more discussion about why
  this model was chosen and which alternatives were considered.~~
\item
  Since this work focused on disentangling specific and unspecific
  priming, it would be really nice to see the relative significance of
  specific and unspecific effects, especially in a figure.
\item
  The authors mention the applicability of this study to the
  experimental assessment of vaccines, but the connection feels a little
  tenuous because vaccines are generally applied to vertebrates and
  vertebrate and invertebrates have very different immune responses. The
  authors may consider discussing these differences in more detail and
  providing a few concrete examples of how this study relates to vaccine
  assessment.~
\item
  The authors state that this study is not concerned with the molecular
  methods involved in immune priming, but it may still be beneficial to
  discuss molecular mechanisms in relation to the results found here.
  Perhaps the mechanisms for immune priming in invertebrates are not
  specific enough to differentiate between strains and would therefore
  support the results found in this study.~
\item
  The authors refer to Ben-Ami et al., 2008, 2010 as ``we''. Please
  consider that this might not be common practice.
\item
  It would be helpful if the authors discuss future directions for this
  work. We are really interested to see where they feel this work should
  go next!~
\end{itemize}

\par\null

We thoroughly enjoyed reading and discussing this preprint in our
journal club and thank the authors for posting their work on the
bioRxiv. We~ sincerely hope that our comments are helpful and we look
forward to seeing the final published version!

\par\null

Best wishes,

The OIST Ecology and Evolution Preprint Journal Club~

\par\null

\selectlanguage{english}
\FloatBarrier
\end{document}

