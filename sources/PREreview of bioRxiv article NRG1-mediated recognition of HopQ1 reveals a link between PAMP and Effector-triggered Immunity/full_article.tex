\documentclass[10pt]{article}

\usepackage{fullpage}
\usepackage{setspace}
\usepackage{parskip}
\usepackage{titlesec}
\usepackage{placeins}
\usepackage{xcolor}
\usepackage{breakcites}
\usepackage{lineno}





\PassOptionsToPackage{hyphens}{url}
\usepackage[colorlinks = true,
            linkcolor = blue,
            urlcolor  = blue,
            citecolor = blue,
            anchorcolor = blue]{hyperref}
\usepackage{etoolbox}
\makeatletter
\patchcmd\@combinedblfloats{\box\@outputbox}{\unvbox\@outputbox}{}{%
  \errmessage{\noexpand\@combinedblfloats could not be patched}%
}%
\makeatother


\usepackage[round]{natbib}
\let\cite\citep




\renewenvironment{abstract}
  {{\bfseries\noindent{\abstractname}\par\nobreak}\footnotesize}
  {\bigskip}

\renewenvironment{quote}
  {\begin{tabular}{|p{13cm}}}
  {\end{tabular}}

\titlespacing{\section}{0pt}{*3}{*1}
\titlespacing{\subsection}{0pt}{*2}{*0.5}
\titlespacing{\subsubsection}{0pt}{*1.5}{0pt}


\usepackage{authblk}


\usepackage{graphicx}
\usepackage[space]{grffile}
\usepackage{latexsym}
\usepackage{textcomp}
\usepackage{longtable}
\usepackage{tabulary}
\usepackage{booktabs,array,multirow}
\usepackage{amsfonts,amsmath,amssymb}
\providecommand\citet{\cite}
\providecommand\citep{\cite}
\providecommand\citealt{\cite}
% You can conditionalize code for latexml or normal latex using this.
\newif\iflatexml\latexmlfalse
\providecommand{\tightlist}{\setlength{\itemsep}{0pt}\setlength{\parskip}{0pt}}%

\AtBeginDocument{\DeclareGraphicsExtensions{.pdf,.PDF,.eps,.EPS,.png,.PNG,.tif,.TIF,.jpg,.JPG,.jpeg,.JPEG}}

\usepackage[utf8]{inputenc}
\usepackage[english]{babel}








\begin{document}

\title{PREreview of bioRxiv article ``NRG1-mediated recognition of HopQ1
reveals a link between PAMP and Effector-triggered Immunity''}



\author[1]{Sophien Kamoun}%
\affil[1]{The Sainsbury Laboratory}%


\vspace{-1em}



  \date{\today}


\begingroup
\let\center\flushleft
\let\endcenter\endflushleft
\maketitle
\endgroup





\selectlanguage{english}
\begin{abstract}
This is a review of Brendolise et al. bioRxiv 293050; doi:
\url{https://doi.org/10.1101/293050}~posted on April 1, 2018. This paper
adds to a current body of research detailing the resistance mechanism
triggered by the~\emph{Pseudomonas~syringe pv. tomato~}effector HopQ1 in
the model plant~\emph{Nicotiana benthamiana}. This plant can be used as
a source of novel disease resistance genes against plant pathogens.%
\end{abstract}%




\par\null

\section*{Summary}

{\label{216049}}\par\null

This paper adds to a current body of research detailing the resistance
mechanism triggered by the~\emph{Pto~}effector HopQ1 in~\emph{N.
benthamiana}; a 2017 paper identified the cognate resistance gene,
Roq1~\cite{Schultink2017}, and a very recent publication showed that the
CCR-NB-LRR (RNL) NRG1 is a required downstream component to mediate the
response~\cite{Qi_2018}.

This paper builds on the authors' previously published method to
identify candidate R genes in~\emph{N. benthamiana}for a given target
effector. In this paper, they employ this method to determine the R
gene(s) required to confer recognition to the~\emph{Pto}effector HopQ1,
which ``is the sole effector {[}from~\emph{Pto}{]}\ldots{} recognized in
both~\emph{Nb~}and\emph{Nicotiana tabacum}'' (lines 96- 97). They argue
that the widespread conservation of HopQ1 among phytopathogenic bacteria
means that, ``the identification of R proteins able to recognize
{[}this{]} effector is potentially valuable to achieve resistance to a
wide range of plant pathogens'' (lines 106- 107).~

The authors show that (1) the RNL NRG1 is required for HopQ1-mediated
cell death in~\emph{N. benthamiana~}and that NRG1 silencing in~\emph{N.
benthamiana} compromises the resistance to DC3000. They also show
evidence that (2) expression of NRG1 within the normally susceptible
plant~\emph{Arabidopsis thaliana~}is sufficient to confer
HopQ1-triggered cell death, as well as confer resistance to DC3000. In
addition, the authors present evidence that (3) NRG1 is induced at an
early time-point following PTI activation and hypothesize that this
represents a link between PTI and ETI.~

The first (1) finding of this paper is convincing, although shown in a
previous publication (Qi et al 2018). The results from the second (2)
line of experimentation are more preliminary, with open questions
stemming from a lack of adequate controls. The final (3) finding of this
paper is underdeveloped, and further experimental validation is required
to suggest that NRG1 represents, ``a link between PAMP and
effector-triggered immunity'' as stated in the title.~~

\section*{\(\)Findings and
comments}

{\label{457509}}\par\null

\textbf{(1)~~NRG1 is required for HopQ1 recognition in~\emph{N.
benthamiana}(Nb) and NRG1 silencing promotes~\emph{Pto~}proliferation in
Nb:~}Agro-mediated silencing of NRG1 in Nb reduces HopQ1-triggered HR;
VIGS-mediated silencing of NRG1 in Nb promotes~\emph{Pto}proliferation~

o~~At no point is the resistance gene Roq1 mentioned in these sections,
even though it is known that HopQ1 is directly recognized by Roq1
in~\emph{N. benthamiana~}(Schultink et al 2017). As written, it is not
clear how the authors propose that NRG1 functions in mediating the cell
death response triggered by Roq1 recognition of HopQ1, or how NRG1
functions in restricting~\emph{Pto}proliferation.~

o~~Previous publication reporting similar findings is not referenced (Qi
et al 2018)

o~~Figure 1a-b: It would be more robust to include quantification of
reps. An explanation of what hp\#26, hp\#27, u135, u111 are targeting
would be helpful the figure legend.~~

o~~There appears to be mis-referencing of figure 1 in this section, e.g.
line 232- 233 should refer to fig 1c rather than 1d.~

\textbf{(2)~~Expression of NRG1 in~\emph{A. thaliana}(At) confers
recognition to HopQ1 and restricts~\emph{Pto}growth:~}HopQ1 and GFP
reporters were transiently expressed in 35s:NbNRG1 transgenic At lines
using biolistic delivery, and HR was evaluated by quantifying a
reduction of the GFP signal resulting from cell death; found a greater
reduction in GFP signal in 3/5 NRG1 lines, compared to GUS lines; in
these 3 lines,~\emph{Pto}proliferation was also reduced in infection
assays

o~~The robustness of the method used to assay the cell death response is
unclear. There is inherent variability in the system as demonstrated by
the differences in GFP levels in the EV control, so can it be claimed
that HopQ1 has less GFP expression due to the activation of HR? A
reduction in GFP over time may serve as a better internal control than a
comparative approach.~

o~~As in the previous sections, no mention of Roq1. It's mentioned in
the discussion that At may have a functional homolog of Roq1, but that
should be stated here and explored in some manner, otherwise these
results are difficult to interpret.~

o~~Figure 3. X axis labels are unclear, use clearer names to distinguish
transgenic lines.~

o~~The conclusion that NRG1 expression in At induces recognition of
HopQ1 to restrict DC3000 growth is not justified from the experiments
shown because (1) NRG1 has been implicated in the function of multiple R
genes and (2) overexpression can cause HR. Better controls are required
to make this claim, such as assaying a DC3000 strain lacking the HopQ1
effector to determine if NRG1 affects other R protein / effector
interactions within At.

o~~Although mentioned in both the introduction and discussion, the
authors don't explore whether HopQ1 and XopQ, a homolog
from~\emph{Xanthomonas}, are recognized by a shared resistance
mechanism. Either XopQ should be downplayed in these sections, or the
resistance mechanism trigged by this effector should be experimentally
explored.

\textbf{(3)~~NRG1 expression is induced at the early stages of the
bacterial infection}: (1) with biolistic delivery system of HopQ1 in Nb,
the authors found no reduction of GFP signal (whereas a reduction in At
was observed); when leaves were pre-treated with Agro, a significant
reduction of GFP signal was measured; (2) when Nb leaves were
infiltrated with Flg22 peptide, NRG1 expression was induced compared to
NRG2 (N-terminally truncated) control

o~~The effect of agro could be explained by technical problems with the
experiment rather than the activation of PTI, particularly as this
result wasn't observed in Arabidopsis.

o~~It cannot be concluded from these experiments that PTI activation is
required for the function of NRG1. Agro infiltration could have multiple
effects on the plant, thus more controls are required. For example, is
there the same effect with flg22 treatment +/- FLS2?

\par\null

o~~The expression analysis lacks proper controls, e.g. unrelated genes
not involved in PTI or functional R genes that do not require PTI
priming to induce HR. Moreover, the expression levels of NRG1 upon flg22
treatment do not necessarily reflect the relative abundance of this
protein within the cell. ~\(\)

\section*{Specific comments}

{\label{687807}}\par\null

-~~~~~~Lines 42-43: ``Propose a model based on the dual requirement of a
CNL and a TNL that could extend beyond HopQ1 detection\ldots{}'' This
model has been proposed previously, and the appropriate publications
should be referenced.

-~~~~~~Line 83- 84: ``resistance from non-host species comprises various
mechanism''; this is quite vague. The idea that non-host resistance
could confer more durable resistance could be elaborated on as this
appears to be the rationale behind the experimental design.~

-~~~~~~Lines 257- 260: ``However, the expression of HopQ1 in Arabidopsis
does not trigger any HR, suggesting that AtNRG1.1 and AtNRG1.2 are not
fully functional homologs of NRG1 and/or able to recognize HopQ1.'' This
seems misleading as it is Roq1, not NRG1, that has been implicated in
direct HopQ1 recognition.

-~~~~~~Lines 280- 281: on the At-NRG1 lines, ``These results suggest
that NRG1 expression in Arabidopsis induces recognition of
HopQ1\ldots{}'' `Induces' takes on a different meaning in this context;
use ``leads to,'' which has a more indirect connotation that better fits
with the data.~

-~~~~~~Line 364- 366: this should be brought up earlier in the
manuscript.~

-~~~~~~NRG1 is referred to as both a CNL and an RNL, please be
consistent with the terminology used.~

\par\null

\section*{Reviewers}

{\label{924286}}\par\null

Erin K Zess, Jessica L Upson, and Sophien Kamoun. The Sainsbury
Laboratory, Norwich Research Park, Norwich, UK.

\selectlanguage{english}
\FloatBarrier
\bibliographystyle{plainnat}
\bibliography{bibliography/converted_to_latex.bib%
}

\end{document}

