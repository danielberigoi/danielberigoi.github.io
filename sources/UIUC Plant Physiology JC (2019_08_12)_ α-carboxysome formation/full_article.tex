\documentclass[10pt]{article}

\usepackage{fullpage}
\usepackage{setspace}
\usepackage{parskip}
\usepackage{titlesec}
\usepackage{placeins}
\usepackage{xcolor}
\usepackage{breakcites}
\usepackage{lineno}





\PassOptionsToPackage{hyphens}{url}
\usepackage[colorlinks = true,
            linkcolor = blue,
            urlcolor  = blue,
            citecolor = blue,
            anchorcolor = blue]{hyperref}
\usepackage{etoolbox}
\makeatletter
\patchcmd\@combinedblfloats{\box\@outputbox}{\unvbox\@outputbox}{}{%
  \errmessage{\noexpand\@combinedblfloats could not be patched}%
}%
\makeatother


\usepackage{natbib}




\renewenvironment{abstract}
  {{\bfseries\noindent{\abstractname}\par\nobreak}\footnotesize}
  {\bigskip}

\renewenvironment{quote}
  {\begin{tabular}{|p{13cm}}}
  {\end{tabular}}

\titlespacing{\section}{0pt}{*3}{*1}
\titlespacing{\subsection}{0pt}{*2}{*0.5}
\titlespacing{\subsubsection}{0pt}{*1.5}{0pt}


\usepackage{authblk}


\usepackage{graphicx}
\usepackage[space]{grffile}
\usepackage{latexsym}
\usepackage{textcomp}
\usepackage{longtable}
\usepackage{tabulary}
\usepackage{booktabs,array,multirow}
\usepackage{amsfonts,amsmath,amssymb}
\providecommand\citet{\cite}
\providecommand\citep{\cite}
\providecommand\citealt{\cite}
% You can conditionalize code for latexml or normal latex using this.
\newif\iflatexml\latexmlfalse
\providecommand{\tightlist}{\setlength{\itemsep}{0pt}\setlength{\parskip}{0pt}}%

\AtBeginDocument{\DeclareGraphicsExtensions{.pdf,.PDF,.eps,.EPS,.png,.PNG,.tif,.TIF,.jpg,.JPG,.jpeg,.JPEG}}

\usepackage[utf8]{inputenc}
\usepackage[greek,english]{babel}



\usepackage[section]{placeins}
\usepackage{float}






\begin{document}

\title{UIUC Plant Physiology JC (2019/08/12): \selectlanguage{greek}α-\selectlanguage{english}carboxysome formation~~~~\selectlanguage{english}}



\author[1]{Steven Burgess}%
\affil[1]{University of Illinois at Urbana–Champaign}%


\vspace{-1em}



  \date{\today}


\begingroup
\let\center\flushleft
\let\endcenter\endflushleft
\maketitle
\endgroup









The UIUC Plant Physiology journal club reviewed the preprint
``\selectlanguage{greek}\textbf{α-carboxysome formation is mediated by the multivalent and
disordered protein CsoS2}\selectlanguage{english}'' (doi: \url{https://doi.org/10.1101/708164})
by Oltrogge et al. 2019. The paper describes the biochemical
characterization of the CsoS2 protein involved in carboxysome assembly,
identifying a repeat peptide region that makes weak electrostatic
interactions with rubisco through the use of bio-layer interferometry
(BLI) and x-ray crystallography. The authors identified evolutionary
conserved residues through protein sequence comparisons and the sites of
interaction between these residues and rubisco through protein X-ray
crystallography.~~

\par\null

We found the paper to be very well written, well presented and valuable
addition to knowledge about alpha carboxysome assembly. Our journal club
assessed the paper as part of a learning exercise about how to make work
accessible to a wide audience.~

\par\null

Participants first learned about the ``and-but-therefore (ABT)'' model
of paper writing popularized by Randy Olsen in his freely available
book~
"\href{https://the-eye.eu/public//concen.org/UChicagoPress.Ebook.Pack-2016-PHC/9780226270708.UChicago\%20Press.Houston\%2C\%20We\%20Have\%20a\%20Narrative_\%20Why\%20Science\%20Needs\%20Story.Randy\%20Olson.Sep\%2C2015.pdf}{Huston,
we have a narrative}``,~ that can be used throughout the manuscript to
help maintain the reader's interest. Focusing on the abstract we found
it to contain many of aspects of the ABT model. We also thought it could
potentially be strengthened by including a stronger ``but'' phrase which
generally represents the question under consideration. It was suggested
that this phrase would start with the fact that there is little
knowledge about how the carboxysome is assembled, and some members of
the club questioned if the ongoing carboxysome engineering efforts might
be mentioned as relevant to the wider importance of the work (either in
the abstract or the discussion).

\par\null

One aspect we found particularly interesting was the similarities
between CsoS2 and the algal protein EPYC which has been implicated in
aggregation of rubisco in the pyrenoid. These appeared to us to an
important point, and the reason to include information about CsoS2 as an
intrinsically disordered protein (IDP) that could perhaps be emphasized
more. As we were not familiar with the PONDR-FIT disorder score, we
would have found it helpful to have a little more explanation as to its
importance and interpretation. Overall we liked the approach for
analyzing IDPs and thought it was an impressive effort to successfully
crystallize the CsoS2 peptide with rubisco.~~

\par\null

In addition, we assessed the presentation of figures, we particularly
liked the use of consistent colouring throughout, the choice of clearly
legible font sizes on all graphs and the helpful diagrams to illustrate
biochemical procedures, such as the BLI procedure in Fig 2b. One
consideration is whether the choice of colors is colorblind friendly,
using the app color oracle, several of the colors are indistinguishable
in all the figures analysed. We also thought inclusion of legend titles
would help guide readers on how best to interpret the data.

\par\null

We thought the X-ray crystallography data was presented in a clear and
helpful manner, displaying what the individual residue interactions were
between the bpeptide and rubisco. If it could be improved further it may
be by inclusion of a label of rubisco for non-experts who may not
immediately associate CbbL and CbbS as subunits. Finally, we
particularly liked Figure 5 as it neatly summarized the proposed role of
CsoS2 in carboxysome assembly.

\par\null

Other thoughts included:

\begin{itemize}
\tightlist
\item
  It would be interesting to include discussion of why the full length
  CsoS2 peptide does not appear to bind rubisco.
\item
  The paper tied up loose ends and did a good job of using multiple
  approaches to build evidence for the direct interaction of CsoS2 and
  rubisco.
\end{itemize}

\par\null

\selectlanguage{english}
\FloatBarrier
\end{document}

