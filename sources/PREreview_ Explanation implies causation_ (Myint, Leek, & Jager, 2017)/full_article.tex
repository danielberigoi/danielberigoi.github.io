\documentclass[10pt]{article}

\usepackage{fullpage}
\usepackage{setspace}
\usepackage{parskip}
\usepackage{titlesec}
\usepackage{placeins}
\usepackage{xcolor}
\usepackage{breakcites}
\usepackage{lineno}





\PassOptionsToPackage{hyphens}{url}
\usepackage[colorlinks = true,
            linkcolor = blue,
            urlcolor  = blue,
            citecolor = blue,
            anchorcolor = blue]{hyperref}
\usepackage{etoolbox}
\makeatletter
\patchcmd\@combinedblfloats{\box\@outputbox}{\unvbox\@outputbox}{}{%
  \errmessage{\noexpand\@combinedblfloats could not be patched}%
}%
\makeatother


\usepackage[round]{natbib}
\let\cite\citep




\renewenvironment{abstract}
  {{\bfseries\noindent{\abstractname}\par\nobreak}\footnotesize}
  {\bigskip}

\renewenvironment{quote}
  {\begin{tabular}{|p{13cm}}}
  {\end{tabular}}

\titlespacing{\section}{0pt}{*3}{*1}
\titlespacing{\subsection}{0pt}{*2}{*0.5}
\titlespacing{\subsubsection}{0pt}{*1.5}{0pt}


\usepackage{authblk}


\usepackage{graphicx}
\usepackage[space]{grffile}
\usepackage{latexsym}
\usepackage{textcomp}
\usepackage{longtable}
\usepackage{tabulary}
\usepackage{booktabs,array,multirow}
\usepackage{amsfonts,amsmath,amssymb}
\providecommand\citet{\cite}
\providecommand\citep{\cite}
\providecommand\citealt{\cite}
% You can conditionalize code for latexml or normal latex using this.
\newif\iflatexml\latexmlfalse
\providecommand{\tightlist}{\setlength{\itemsep}{0pt}\setlength{\parskip}{0pt}}%

\AtBeginDocument{\DeclareGraphicsExtensions{.pdf,.PDF,.eps,.EPS,.png,.PNG,.tif,.TIF,.jpg,.JPG,.jpeg,.JPEG}}

\usepackage[utf8]{inputenc}
\usepackage[english]{babel}








\begin{document}

\title{PREreview: Explanation implies causation? (Myint, Leek, \& Jager, 2017)}



\author[1]{C.H.J. Hartgerink}%
\author[2]{Laura Kunst}%
\author[2]{l.h.jutten}%
\affil[1]{Tilburg University}%
\affil[2]{Affiliation not available}%


\vspace{-1em}



  \date{\today}


\begingroup
\let\center\flushleft
\let\endcenter\endflushleft
\maketitle
\endgroup





\selectlanguage{english}
\begin{abstract}
A preprint review of ``Explanation implies causation?'' by Myint, Leek,
\& Jager. Posted on bioRxiv, November 13 2017. doi:
\href{http://doi.org/10.1101/218784}{10.1101/218784}%
\end{abstract}%




Thank you for posting this preprint on bioRxiv! We had fun reading and
discussing it. In order to structure our own reading of the paper and to
not have comments disappear in the void, we compiled some of our
comments for your consideration. We hope you find this feedback helpful.
We also easily and happily reproduced all your tables using the provided
data and code!

\emph{What is the main question the study attempts to answer?}

The study presented in this manuscript investigates whether causal
language in the reporting of results obfuscates the methods, goal and
scope of the analysis presented. In other words: Can people distinguish
between a study whose goal was inferential and one whose goal was
actually causal, regardless of the language results are presented in?

\emph{What is (are) the hypothesis(es)?}

The authors hypothesized a main effect of causal language on the scope
of the analysis (page 2), such that inferential findings presented in
causal language would be more often considered causal. Moreover, the
authors hypothesized an interaction, where being presented with a
``causal'' answer option would show stronger scope extension than when
this answer option was not available.

\emph{What techniques/analyses do the researchers adopt to test their
hypothesis(es)?}

The authors apply an experimental design in an online course, using a
vignette about the link between smoking and lung cancer (levels: with v
without causal language). Additionally, participants were either
presented with causal as one of four multiple choice options, or not. As
such, the design was a 2 by 2 factorial design. Students were asked to
indicate the type of analysis conducted and the dependent measure is the
proportion of students with the correct answer (i.e., inferential
analysis)

\emph{Why is this study relevant?}

Published empirical research papers oversell results and exaggerate
claims by using inappropriate and out of scope statements. Rises in
``inappropriate causal language'' are mentioned for a few fields, such
as nutritional and educational fields. Considering that humans seek
order in the world, it seems plausible that this also happens in the
interpretation of research findings, but that researchers might forget
to keep these claims tentative. If such language affects the interpreted
scope of the results, it is important as a self-reflection in order to
adjust writing behavior in the future to better reflect the scope of the
analysis.~

\emph{Write here any general comments you might have about the research
approach.}

We have several suggestions for improvement that we would like to offer
the authors. First we present several general and methodological points,
and we then move on to discuss the rationale and relevance of the study.
~

Methodologically, first, the used vignette in the study is on smoking
and lung cancer. We wonder whether using this example, which has a
strong social/historical/cultural context of being causal, is
appropriate for the point the authors are trying to make. The
explanation might prime this and create stronger effects than other
associations. We wonder whether using a different vignette would have
produced different results (e.g., the relation between anxiety and
depression or between drug abuse and impulsivity, which have less clear
causal directions, or even the association between, say,~the number of
storks in a given country and its correlation with the number of babies
born).~

Second, in most scientific papers, causal interpretations~of
correlational studies are accompanied by `warnings' that~additional
studies are needed to clarify the causality in a certain relationship,
or that~a limitation of the study~is its correlational character.

Third, the vignette uses more expressive language than we anecdotally
recall (``We explain'' more often seems to be presented along the lines
of ``We expect'').~This may exaggerate the problems of language.~

In order to increase the external validity of the vignette, we suggest
to create a different vignette (point one); create a condition in which
corrective information ('warnings') is added versus not added (point
two) and (point three) use fewer exaggerations while doing so.~~

Fourth, the authors do not mention how other aspects of the scholarly
communication system also contribute to~ the overreaching in the
interpretation of findings. For example, high-rejection rates at
journals make editors select articles with ``hot'' headlines and
researchers' degrees of freedom stretch the evidence gathering process
and potentially also change the scope of the analysis. Including these
aspects in the (introduction of the) study would heighten the study's
relevance.

Regarding the relevance and rationale of the~study,~the authors~describe
in their introduction~two `problems'~that may result from interpreting
correlational findings as causal. They (a) list several examples of
newspapers and other media outlets, which include obvious (and amusing)
erroneous interpretations and (b) note~that also within research, errors
are~made in interpreting correlational evidence as causal. The issue
here is that it is unclear which of these `problems' (scenario a or b)
the authors focus on, and therefore what the study results~implicate.

Our impression is that the authors want to focus on scenario b,~thus,
describing potential causal explanations for correlational findings in
research as `wrong', because it suggests to other researchers that the
study was actually of causal nature. This is of course a challenging
statement that would give rise to an interesting discussion, regarding
researchers' habits of providing potential causal explanations for their
findings. There are however some issues in the manuscript that lead us
to think that such a claim is not supported.

First, the introduction states that ``inappropriate causal language''
has been found in a few fields, such as nutritional and educational
fields. However, to our anecdotalknowledge, most articles appropriately
describe in their methods and discussion sections~that an inferential
approach has been used, and that one must be careful not to interpret
the result as causal. In many research schools it is viewed as essential
to provide some theoretical framework and possible (causal)
interpretation of the current results, accompanied by a statement
clarifying that the current data are correlational. If the authors argue
that this is not happening correctly in scientific practice, more
evidence for this argument should be brought forward in the
introduction. Alternatively, the authors should specify to which
specific scientific fields the results apply.

Second, the main audience for scientific articles consists of scientists
or students with moderate~knowledge on statistical methods. It seems
inefficient to ask of researchers that they~refrain from describing any
theoretical framework that might suggest a causal relationship, because
individuals without sufficient statistical and methodological knowledge
might read the papers. It seems reasonable that scientific papers should
mostly be comprehensible for other scientists and students with some
knowledge on methods and statistics. Given that students following an
introductory class in statistics were~the~study population, the claim
that researchers in general are interpreting correlational inferential
data erroneously, seems unsupported.~

This brings us to the issue of generalizability. Do you view the
students as representative of researchers, the press, or the general
population? The study uses a causal design by using a randomized
experiment, but has a non-random sample and does not allow for
inferences. One could even consider the student ``sample'' as being the
population and deem inferences beyond this not applicable.

Nevertheless, the authors bring up an important issue in the
introduction regarding erroneous interpretations of the media (named
scenario~b above). We suggest that your paper has more clear-cut
implications for the extent to which readers who are potentially not
well-versed in research and statistics (given they are students),
interpret research findings as causal versus correlational, than the
researcher population at large. This point of~view may better match
your~rationale in the introduction and the examples you have included.
Moreover, the students you have tested may be more representative of
people-who-are-not-(yet)-well-schooled-in-statistics, such as
individuals working for newspapers, than of researchers.~

Finally, regardless of which focus the authors choose, we note that the
paper shows there is a problem of how strongly expressive language can
affect student's interpretation of findings, but provides little
framework for how to move forward. For example, it is implied throughout
the paper that it is a bad thing to stretch the scope of an inferential
analysis to causal (with which we agree), but does not show how we might
prevent it. The manuscript seems to imply that~researchers should not
indicate the hypotheses underlying a certain correlation at all. If this
is not the opinion of the authors, we recommend sharpening the writing
to clarify what is and what is not permitted from your perspective and
how to move forward.

\emph{Specific comments about experimental approaches and methods used
in the study}:

\begin{itemize}
\tightlist
\item
  It is unclear where the 68.5\% comes from {[}page 3{]}. Upon
  reproducing, we understood where it comes from but it distracted us
  during the reading.
\item
  During the reading it was really difficult to understand how 20,256
  students resulted in approximately 11,000 included in the tables. If
  this is due to the exclusion of re-takers of the test, then please
  indicate the difference in the text.
\end{itemize}

\par\null

\emph{Sspecific commenst/notes about figures in the paper :}

We really enjoy Table 1, and the CC BY helps for reuse in classes. What
we did during the reading of the paper was draw out the percentage
correct in a simple plot, to visualize how the conditions relate to each
other. This also facilitates the interpretation in light of the
hypotheses, which seem to imply a main effect and an interaction. We
think it would be helpful if you would add such a visualization, because
in our experience the interpretation is easier to discern visually.~

\emph{Any additional comments:}

\begin{itemize}
\tightlist
\item
  ``We explain, we think'' {[}page 3{]}
\item
  ~``The size of the effect is must smaller'' incorrect? {[}page 4{]}
\item
  ~'To confirm our results.. ' is it really a confirmation, and is this
  the reason its done? {[}page 5{]}
\item
  ~'the data is visualized~ \ldots{} ' --\textgreater{} data are
  visualized {[}page 5{]}
\item
  GitHub is not a sustainable hosting service --- you could archive the
  contents for preservation at Zenodo via the GitHub release link.
\end{itemize}

\selectlanguage{english}
\FloatBarrier
\end{document}

