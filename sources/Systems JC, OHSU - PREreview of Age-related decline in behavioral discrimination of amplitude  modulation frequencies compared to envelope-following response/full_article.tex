\documentclass[10pt]{article}

\usepackage{fullpage}
\usepackage{setspace}
\usepackage{parskip}
\usepackage{titlesec}
\usepackage{placeins}
\usepackage{xcolor}
\usepackage{breakcites}
\usepackage{lineno}





\PassOptionsToPackage{hyphens}{url}
\usepackage[colorlinks = true,
            linkcolor = blue,
            urlcolor  = blue,
            citecolor = blue,
            anchorcolor = blue]{hyperref}
\usepackage{etoolbox}
\makeatletter
\patchcmd\@combinedblfloats{\box\@outputbox}{\unvbox\@outputbox}{}{%
  \errmessage{\noexpand\@combinedblfloats could not be patched}%
}%
\makeatother


\usepackage[round]{natbib}
\let\cite\citep




\renewenvironment{abstract}
  {{\bfseries\noindent{\abstractname}\par\nobreak}\footnotesize}
  {\bigskip}

\renewenvironment{quote}
  {\begin{tabular}{|p{13cm}}}
  {\end{tabular}}

\titlespacing{\section}{0pt}{*3}{*1}
\titlespacing{\subsection}{0pt}{*2}{*0.5}
\titlespacing{\subsubsection}{0pt}{*1.5}{0pt}


\usepackage{authblk}


\usepackage{graphicx}
\usepackage[space]{grffile}
\usepackage{latexsym}
\usepackage{textcomp}
\usepackage{longtable}
\usepackage{tabulary}
\usepackage{booktabs,array,multirow}
\usepackage{amsfonts,amsmath,amssymb}
\providecommand\citet{\cite}
\providecommand\citep{\cite}
\providecommand\citealt{\cite}
% You can conditionalize code for latexml or normal latex using this.
\newif\iflatexml\latexmlfalse
\providecommand{\tightlist}{\setlength{\itemsep}{0pt}\setlength{\parskip}{0pt}}%

\AtBeginDocument{\DeclareGraphicsExtensions{.pdf,.PDF,.eps,.EPS,.png,.PNG,.tif,.TIF,.jpg,.JPG,.jpeg,.JPEG}}

\usepackage[utf8]{inputenc}
\usepackage[english]{babel}








\begin{document}

\title{Systems JC, OHSU - PREreview of ``Age-related decline in behavioral
discrimination of amplitude~ modulation frequencies compared to
envelope-following responses''}



\author[1]{Daniela Saderi, Ph.D.}%
\affil[1]{Oregon Health \& Science University}%


\vspace{-1em}



  \date{\today}


\begingroup
\let\center\flushleft
\let\endcenter\endflushleft
\maketitle
\endgroup





\selectlanguage{english}
\begin{abstract}
This is a preprint journal club review of \textbf{``Age-related decline
in behavioral discrimination of amplitude modulation frequencies
compared to envelope-following responses''~}by Jesyn Lay, Edward L.
Bartlett. The preprint was posted on bioRxiv on Sep 28, 2017 (DOI:
\url{http://dx.doi.org/10.1101/193268}).

\emph{\href{https://hearingbrain.org/systemsjournalclub.php}{Our group}
reviewed this paper on October 13, 2017 and used the PREreview quick
participant worksheet.}%
\end{abstract}%




\selectlanguage{english}
\begin{figure}[h!]
\begin{center}
\includegraphics[width=0.28\columnwidth]{figures/lbhb-logo-small/lbhb-logo-small}
\caption{{\href{https://hearingbrain.org/systemsjournalclub.php}{Systems
Neuroscience JC} - Oregon Health \& Science University
{\label{267279}}%
}}
\end{center}
\end{figure}

Dear Authors,

Thank you for posting this manuscript as a preprint on bioRxiv! We
reviewed this work at our Systems Neuroscience journal club at OHSU.
Compiled comments from the attendants are below. To structure the
feedback used the quick worksheet guidelines published on PREreview. We
hope this feedback will be useful to improve the manuscript.

\emph{What is the main question the study attempts to answer?}

This study attempts to establish a relation between significant decline
of auditory neural responses measured by envelope following responses
(EFRs) and perceptual deficits in detecting fast amplitude-modulated
frequency sounds associated with age-related hearing loss in rats.

\emph{What is (are) the hypothesis(es)?}

The authors hypothesized that the well know decline in EFR in older rats
in response to faster amplitude-modulated frequencies (AMF) is
accompanied by a deficit in AM perception of those sounds.~~

\emph{Why is this study relevant?}

Human speech contains fast modulations in amplitude and frequency that
are crucial for speech intelligibility. Age-related hearing loss in
humans is associated with difficulty in understanding speech often as a
consequence of poor temporal modulation coding. Understanding if the
lower electrophysiological responses to fast AMF measured in aged rats
are associated with behavioral deficits in the same animals is important
to establish the link between humans and animal models of age-related
hearing loss.

\emph{What techniques/analyses do the researchers adopt to test their
hypothesis(es)?}

Behavioral tests:

\begin{enumerate}
\tightlist
\item
  Acoustic startle response (ASR) and inhibition of the startle reaction
  using a prepulse (prepulse inhibition, PPI). The magnitude of the PPI
  is proportional to the animal's sound detectability such as the
  highest the PPI response, the \ldots{}
\item
  Pure tone detection behavior.
\item
  AMF discrimination task
\end{enumerate}

Electrophysiological measurements

\begin{enumerate}
\tightlist
\item
  Auditory evoked potentials as auditory brainstem responses (ABRs) and
  envelop following responses (EFRs).
\end{enumerate}

\emph{Write here any specific comment you might have about experimental
approaches and methods used in the study.}

\emph{Write here any general comments you might have about the research
approach.}

\emph{Write here any specific comment/note about figures in the paper
(this could be related to the way data are displayed and your ability to
understand the results just by looking at the figures).}

It would be useful to include a diagram of the behavioral paradigms to
help readers understand how the experiment is set up.~

In \textbf{Figure 1}, to make it clear where you measure the startle
response and to show the actual shape of the measured response you could
use figure 2 waveform of the startle response instead of the square
pulse. We would recommend combining these figures into one figure,
perhaps with two panels.

\textbf{Figure 3}. It would be nice to show the psychometric functions
as it looks like you would get a good one of the ASR magnitude but not
for RMS ratio. This would show clearly that ASR magnitude measurements
are a better way to measure behavioral thresholds (level at which there
is no measurable change in startle response), key information to
establish a link with human perceptual measurements. From now on, you
could avoid showing the RMS ratio measurement in the main figures, and
perhaps have them in the supplementary or report the statistics only in
the text?

\textbf{Table 1} is a bit confusing as it's hard to understand what the
important part is. Could you for instance highlight relative to 75 the
first level that has a significantly lower response?

In \textbf{Figure 4} it would help to label that the change in AM
frequency is related to the prepulse stimulus.

\textbf{Figure 6} is very dense and hard to interpret. What are the
matched waves? It's a bit confusing to understand what's going on
because sometimes the response is even higher in the matched condition.
Would it be useful to visualize the mean of the matched intensity.

\textbf{Figure 7} is very interesting. Even after you equated the
levels, you still got much less PPI in the older animals than in the
younger. This would suggest that there is something happening at higher
levels of auditory processing.

In \textbf{Figure 8} for visualization purposes, it would be better if
the solid solid line were the 100\% modulation depth. Additionally, to
make the figure more clear you could remove the 25\% for young as you
don't have that measurement in all conditions.~

\textbf{Figure 9~}perhaps should be brought up in the preprint because
if left at the end after the references it is not easily accessible to
the reader. This figures attempts to compare behavioral results with
electrophysiological measurements. The message of this figure is that
the behavior of the animal changes more compared to the electorphy
responses. However it is a bit hard to compare a ratio with a linear
scale. Perhaps it would be easier to have one panel that shows the
change with 100\% PPI and another with the change in EFR and compare
between groups rather than comparing between measures?~

One panel with change in 100\% PPI and another with change in EFR and
then you compare between the groups rather than comparing between the
measures? We did not understand what the baseline was. Did you try doing
everything in a multiple linear regression? Does EFR predict PPI on
individual level not for averages. PPI=a+b*EFR.~

The relative change affects the behavior rather than the absolute
repsonse? the 50\% depth and 100\% are similar in younger but
performance looks quite different. It appears that the relative change
in EFR does not predict the behavior. Is that correct?

\emph{Write here any additional comment you might have (this includes
minor concerns such as typos and structure of the manuscript)}

Introduction:

\begin{itemize}
\tightlist
\item
  I wonder if you need to go into such details to understand the circuit
  in order to motivate the study. I would focus on stating the problem,
  what's known about the problem, what is/are you hypothesis/es, and how
  you are going to test it/them. Conclude the intro with a short
  paragraph about your findings. Leave the description of the methods to
  the method session. P
\item
  Perhaps it would be helpful to clarify the distinction between
  previous findings in human and animal models. It could be important to
  report any psychophysical evidence that supports how older humans have
  harder time understanding fast speech. Additionally, you might want to
  mention that the way that the startle response circuit is wired in
  humans is different from the rat.~
\end{itemize}

\par\null\par\null







\textbf{}

\selectlanguage{english}
\FloatBarrier
\end{document}

