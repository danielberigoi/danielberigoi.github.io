\documentclass[10pt]{article}

\usepackage{fullpage}
\usepackage{setspace}
\usepackage{parskip}
\usepackage{titlesec}
\usepackage{placeins}
\usepackage{xcolor}
\usepackage{breakcites}
\usepackage{lineno}





\PassOptionsToPackage{hyphens}{url}
\usepackage[colorlinks = true,
            linkcolor = blue,
            urlcolor  = blue,
            citecolor = blue,
            anchorcolor = blue]{hyperref}
\usepackage{etoolbox}
\makeatletter
\patchcmd\@combinedblfloats{\box\@outputbox}{\unvbox\@outputbox}{}{%
  \errmessage{\noexpand\@combinedblfloats could not be patched}%
}%
\makeatother


\usepackage[round]{natbib}
\let\cite\citep




\renewenvironment{abstract}
  {{\bfseries\noindent{\abstractname}\par\nobreak}\footnotesize}
  {\bigskip}

\renewenvironment{quote}
  {\begin{tabular}{|p{13cm}}}
  {\end{tabular}}

\titlespacing{\section}{0pt}{*3}{*1}
\titlespacing{\subsection}{0pt}{*2}{*0.5}
\titlespacing{\subsubsection}{0pt}{*1.5}{0pt}


\usepackage{authblk}


\usepackage{graphicx}
\usepackage[space]{grffile}
\usepackage{latexsym}
\usepackage{textcomp}
\usepackage{longtable}
\usepackage{tabulary}
\usepackage{booktabs,array,multirow}
\usepackage{amsfonts,amsmath,amssymb}
\providecommand\citet{\cite}
\providecommand\citep{\cite}
\providecommand\citealt{\cite}
% You can conditionalize code for latexml or normal latex using this.
\newif\iflatexml\latexmlfalse
\providecommand{\tightlist}{\setlength{\itemsep}{0pt}\setlength{\parskip}{0pt}}%

\AtBeginDocument{\DeclareGraphicsExtensions{.pdf,.PDF,.eps,.EPS,.png,.PNG,.tif,.TIF,.jpg,.JPG,.jpeg,.JPEG}}

\usepackage[utf8]{inputenc}
\usepackage[english]{babel}








\begin{document}

\title{Review of~Medicago truncatula Zinc-Iron Permease6 provides zinc to
rhizobia-infected nodule cells~DOI: 10.1101/102426 (BioRxiv),
10.1111/pce.13035 (Plant Cell \& Environment)}



\author[1]{Elsbeth Walker}%
\author[1]{Ahmed Ali}%
\author[1]{Rakesh Kumar}%
\author[1]{Miriam Hernandez Romero}%
\author[1]{Maura Zimmermann}%
\author[1]{Gurpal Singh}%
\author[1]{Rebecca Brennan}%
\author[1]{David Chan Rodriguez}%
\author[1]{Stavroula Fili}%
\author[1]{Erin Patterson}%
\author[1]{Ahmet Bakirbas}%
\affil[1]{University of Massachusetts Amherst}%


\vspace{-1em}



  \date{\today}


\begingroup
\let\center\flushleft
\let\endcenter\endflushleft
\maketitle
\endgroup









\section*{\texorpdfstring{\emph{Medicago truncatula} Zinc-Iron Permease6
provides zinc to rhizobia-infected nodule
cells}{Medicago truncatula Zinc-Iron Permease6 provides zinc to rhizobia-infected nodule cells}}

{\label{463319}}

\textbf{{[}Isidro~Abreu,~Angela~Saez,~Rosario~Castro-Rodriguez,~Viviana~Escudero,~Benjamin~Rodriguez-Haas,~
Marta~Senovilla, Camille~Laure, Daniel~Grolimund, Manuel~Tejada-Jimenez,
Juan~Imperial,~Manuel~Gonzalez-Guerrero , January 24, 2017 (preprint),~
September 21, 2017 (in print), BioRxiv \& Wiley-Blackwell{]}}

\textbf{DOI: 10.1101/102426 (BioRxiv), 10.1111/pce.13035 (Plant Cell \&
Environment)}

\par\null

\textbf{Overview}

\par\null

Abreu~\emph{et al.}~have made contributions to the understanding of the
transport of zinc in rhizobia infected nodule cells by~ \emph{Medicago
truncatula~}Zinc-Iron Permease 6 (\emph{MtZIP6,~Medtr4g083570}). Their
results showed that \emph{MtZIP6} is expressed in the differentiation
zone of the nodules and particularly in the plasma membrane of the
infected cells. \emph{MtZIP6} knockdown plants had a significant
reduction of nitrogen fixation activity. Moreover, accumulation of zinc
was detected in the apoplastic regions of the differentiation zone
showing the inability of these plants to transport zinc inside the
infected cells.

\par\null

First two figures were critical in terms of providing evidence to
support authors' initial claims. In Figure 2A authors show that the GUS
reporter activity is seen only in the apical region of the nodule and a
closer look at the longitudinal section of the nodule clearly indicates
the presence of GUS activity in the older part of differentiation zone
(Zone II) and the younger parts of the fixation zone (Zone III).~ The
expression profile of MtZIP6 is also shown to be mostly in Zone II and
Zone III with the help of RNA-seq performed on the laser-capture
microdissected~\emph{M. truncatula} nodule cells isolated from all the
different zones. Figure 3 provides evidence for subcellular localization
of MtZIP6. A composite multi-fluorophore image shows MtZIP6 to be
localized in the plasma membrane.~

Figure 4 and figure 5 compare the symbiosis between control
and~\emph{mtzip6} RNAi line under mineral supply and in absence of
mineral supply respectively.

In figure 4, the authors point out that there is no effect on the growth
between the control and the~\emph{mtzip6~}RNAi line under a
non-symbiotic condition with NH4NO3 supplementation. These figures
clearly help to prove the point that MtZIP6 has some important function
in the development of healthy symbiosis as supported by results about
the number of red(functional) and white(nonfunctional) nodules, dry
weight and Acetylene reduction assay.

\par\null

Figure 6C and D aimed to show the differences in zinc accumulation in a
WT nodule compared to the~\emph{mtzip6} RNAi knockdown. It was stated
that higher accumulation occurred in the differentiation zone of the
mutant nodule. However, the images gave a differing impression. It is
clear that the zinc does not disperse into the rest of the nodule and
indeed remains in the differentiation zone, but a higher zinc
accumulation compared to WT nodule is not clear. Perhaps it should be
noted in the figure legend that figure 6D have been altered in order to
avoid oversaturation in the image. ~~

\par\null

The authors present a proposed model to summarize their findings and
their hypothesis for zinc uptake by nodule cells in figure 7. The model
fits their results, which showed that MtZIP6 is most probably localized
in the cell membrane of the differentiation cells and that knockdown
of~\emph{MtZIP6} accumulate zinc in the apoplastic regions. So,
suggesting that MtZIP6 is responsible for transporting zinc inside cells
infected with rhizobia seems like a plausible conclusion. In this model,
the authors attempt also to fill in other missing pieces of the zinc
uptake pathway in nodules.~

~

\textbf{Positive feedback:}

\par\null

Figure 1A and figure 2 were the key figures where authors had the chance
to support their initial claims on gene expression of~\emph{MtZIP6}~in
nodules and its localization to differentiation zone. Localization
experiments displayed in figure 3 were necessary to support their claims
on the localization of MtZIP6 to the plasma membrane. We appreciated the
three-dimensional reconstruction to~point out the localization more
clearly.~

\par\null

Furthermore, we really liked the transient expression assay in
Agrobacterium infiltrated tobacco leaves to show plasma membrane
localization of MtZIP6. The result was clean and succinct, MtZIP6
overlayed perfectly with plasma membrane localized~CFP. We wish it was
included in~Figure 3 along with other localization experiments.

\par\null

Figure 4 and 5 provided much-needed insight into the function of MtZIP6.
But we think that these two figures can be combined together and this
will help to provide a better view of the overall picture about the
vital need of MtZIP6. This would also have provided space to include the
tobacco results in the paper instead of supplemental material.~

\par\null

Abreu~\emph{et al.} showed nicely in figure 6, the possible role
of~MtZIP6 as zinc transporter in rhizobia-infected nodules. We think it
is very interesting~that knockdown of~\emph{MtZIP6} does not affect zinc
concentration in the plant. This result suggests a nodule-specific
function of the MtZIP6 transporter.~

\par\null

The novelty of MtZIP6 is a notable finding and gives exciting
justification to characterize the protein. Overall the paper and its
findings were easily understood, even of nonexperts.~

\par\null

\textbf{Major concerns:}

\par\null

The title of this paper accurately reflects the results in most cases,
alas, some results raised questions after reading the paper. In our
first reading, the necessity of figure 1B was not clearly understood. It
would be helpful to emphasize how unrelated MtZIP6 and GmZIP1 are from
each other. They are highlighted in red in the figure, however, they are
not commented in the figure legend. These are homologs with similar
functions, yet they do not group together in a tree based on sequence
similarity. Based on the evidence in figure 3, authors propose that
MtZIP6-HA is localized in the stele. We think this was a very broad area
to define localization of MtZIP6. We appreciated that this proposed
localization was changed in the print version of the paper.~

\par\null

We had concerns that Figures 4 and 5 could have been combined into a
single figure. They address the same finding, and side by side
comparison of Figure 4B and 5A, as well as 4C and 5B, may have
emphasized differences and consolidated figure legends. There were
questions about the statistical~significance of findings presented in
Figure 4.

Regarding figure 6B, we are confused about what message the authors want
to send. The figure supports the zinc-dependent concentration of~MtZIP6,
but it was relatively difficult to understand the rationale to test
different concentrations of zinc. Since figure 6A showed that zinc
concentration in WT nodules is lower compared to nodules in RNAi line,
in figure 6C I was expecting to observe the same zinc concentration
pattern shown in figure 6A. We wonder if the exposure times were changed
during image acquisition to prevent oversaturation of the image.

\par\null

In figure 7,~ uncharacterized transporters are denoted by question
marks. We think it is helpful for the reader to be able to look at a
summary model but the question marks in the suggested transporters might
be misleading. After all, it is just a suggested model which summarizes
what we know so far from studies in other species; it is obvious that
further studies are needed to validate it. Some of the not yet
characterized transporters were eliminated in the final publication,
helping to streamline main findings.~~

\selectlanguage{english}
\FloatBarrier
\end{document}

