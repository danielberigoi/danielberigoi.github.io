\documentclass[10pt]{article}

\usepackage{fullpage}
\usepackage{setspace}
\usepackage{parskip}
\usepackage{titlesec}
\usepackage{placeins}
\usepackage{xcolor}
\usepackage{breakcites}
\usepackage{lineno}





\PassOptionsToPackage{hyphens}{url}
\usepackage[colorlinks = true,
            linkcolor = blue,
            urlcolor  = blue,
            citecolor = blue,
            anchorcolor = blue]{hyperref}
\usepackage{etoolbox}
\makeatletter
\patchcmd\@combinedblfloats{\box\@outputbox}{\unvbox\@outputbox}{}{%
  \errmessage{\noexpand\@combinedblfloats could not be patched}%
}%
\makeatother


\usepackage[round]{natbib}
\let\cite\citep




\renewenvironment{abstract}
  {{\bfseries\noindent{\abstractname}\par\nobreak}\footnotesize}
  {\bigskip}

\renewenvironment{quote}
  {\begin{tabular}{|p{13cm}}}
  {\end{tabular}}

\titlespacing{\section}{0pt}{*3}{*1}
\titlespacing{\subsection}{0pt}{*2}{*0.5}
\titlespacing{\subsubsection}{0pt}{*1.5}{0pt}


\usepackage{authblk}


\usepackage{graphicx}
\usepackage[space]{grffile}
\usepackage{latexsym}
\usepackage{textcomp}
\usepackage{longtable}
\usepackage{tabulary}
\usepackage{booktabs,array,multirow}
\usepackage{amsfonts,amsmath,amssymb}
\providecommand\citet{\cite}
\providecommand\citep{\cite}
\providecommand\citealt{\cite}
% You can conditionalize code for latexml or normal latex using this.
\newif\iflatexml\latexmlfalse
\providecommand{\tightlist}{\setlength{\itemsep}{0pt}\setlength{\parskip}{0pt}}%

\AtBeginDocument{\DeclareGraphicsExtensions{.pdf,.PDF,.eps,.EPS,.png,.PNG,.tif,.TIF,.jpg,.JPG,.jpeg,.JPEG}}

\usepackage[utf8]{inputenc}
\usepackage[english]{babel}








\begin{document}

\title{OIST E\&E PREreview JC ``Biodiversity trends are stronger in marine than
terrestrial assemblages''}



\author[1]{maggi brisbin}%
\author[1]{Jigyasa Arora}%
\author[1]{Yuka Suzuki}%
\author[1]{Kun-Lung Li}%
\author[1]{Jigyasa Arora}%
\author[1]{Julian Katzke}%
\affil[1]{Okinawa Institute of Science and Technology Graduate School}%


\vspace{-1em}



  \date{\today}


\begingroup
\let\center\flushleft
\let\endcenter\endflushleft
\maketitle
\endgroup









\section*{Biodiversity trends are stronger in marine than terrestrial
assemblages}

{\label{196514}}

Shane~Blowes,~Sarah~Supp,~Laura~Antao,~Amanda~Bates,~Helge~Bruelheide,~Jonathan~Chase,~Faye~Moyes,~Anne~Magurran,~Brian~McGill,~Isla~Myers-Smith,~Marten~Winter,~Anne~Bjorkman,~Diana~Bowler,~Jarrett
EK~Byrnes,~Andrew~Gonzalez,~Jes~Hines,~Forest~Isbell,~Holly~Jones,~Laetitia~Navarro,~Patrick~Thompson,~Mark~Vellend,~Conor~Waldock,~Maria~Dornelas

bioRxiv, October 30th, 2018

doi:~\url{https://doi.org/10.1101/457424}

\par\null

\textbf{Overview and take-home messages:}

Blowes~et al\emph{.~}tackle an impressive and large undertaking in this
paper by attempting to disentangle global biodiversity trends through a
meta-analysis of data from 358 studies. By dividing the available data
by biome and taxa, the authors were able to detect different
biodiversity trends in marine and terrestrial biomes. Tropical marine
biomes, particularly the Caribbean, have a more negative deviation from
the mean trend in species richness and more positive deviations from the
overall trend in species turnover--species are turning over more quickly
in marine biomes. The analyses demonstrate that mean local species
richness is not decreasing, but many individual regions deviate
significantly from the overall mean. The results have important
implications for how we discuss changes in biodiversity in the
anthropocene, but it is important to make clear that locally static
species richness does not equate to globally static species richness,
and species are going extinct at an alarming rate. Overall, this paper
presents careful analyses and is clearly written, however, there are a
few issues that, if addressed, we feel could improve future versions of
the manuscript.~~

\textbf{Major concerns:}

\begin{itemize}
\tightlist
\item
  The authors make excellent use of a public, curated time-series
  database to facilitate this study. A major limitation in interpreting
  the results is the uneven distribution of data across regions (33
  marine biomes v. 10 terrestrial and 5 freshwater) and the gap in data
  from tropical terrestrial ecosystems. It is feasible that inclusion of
  tropical terrestrial systems would change the mean trend in local
  species richness and eliminate the differential deviance between
  marine and terrestrial biomes from the overall trend. While we
  understand that including tropical terrestrial ecosystems in the
  current analysis is probably not possible, we would appreciate more
  discussion of both the cause and effects of this limitation. Why are
  data unavailable for tropical terrestrial systems? Have they not been
  collected or did they not meet the selection criteria for the
  database? While selection criteria for the database is likely
  discussed in the publication describing its initiation, it may be
  helpful to readers to briefly describe how data was selected for
  inclusion in the database and the current study.
\item
  The authors clearly state that baseline availability and selection is
  a major issue in time-series studies and they did an impressively
  thorough job in their sensitivity analyses examining the effect of
  baseline selection on their results. Regardless, it is extremely
  difficult to compare studies with such varied lengths as 2-95 years
  and the length of the time series should affect the change in
  biodiversity observed. While it would cause an undesired reduction in
  data, we are curious if the authors considered pruning the data to
  include studies that cover a minimum number of years within a
  specified time period (e.g. at least 10 years of observations with at
  least 1 observation before 1980). We are curious how many studies
  would pass such filters and how it would affect the results.
  Alternatively, baselines might be set to the mean richness in all
  observations for a given time-series with changes in richness as the
  difference from the mean rather than the difference from an earlier
  observation. This method has been used to assess the impact of extreme
  events, such as hurricanes, on community compositions.~
\item
  While ecologically fascinating that the mean local species richness is
  not changing over time, it could easily be conflated with global
  species richness remaining unchanged if the paper is not read
  carefully. We are slightly concerned that non-specialist readers,
  including the press, may interpret the results to mean that global
  biodiversity is not decreasing and use the paper in rhetoric
  challenging the reality of climate change. Lines 311-314 are paramount
  to this work, and deserve potentially even more emphasis, but we feel
  some treatment of the difference between the subject of the study
  (changes in local species richness) and global extinction rate could
  be useful. Despite the results presented here, climate change is
  causing species extinction (?). If there are global species
  richness/abundance data available, it could be helpful to include or
  discuss them.
\end{itemize}

\textbf{Minor concerns:}

\begin{itemize}
\tightlist
\item
  The first paragraph of the discussion adeptly summarizes the most
  important and interesting findings of this study. It may be helpful to
  highlight this finding in the title.
\item
  In figures 2 and 4 the backgrounds of the individual density ridges
  are opaque instead of transparent and therefore occlude other data and
  the grey shading for the confidence interval. Figures 2 and 4 may be
  more clear to readers if the biomes in panel a are sorted by
  latitudinal band to match the sorting in panel b. Visually
  distinguishing realms in panel a could also be improved. Several
  members of our group initially thought it was a mistake that the space
  next to ``marine'' in the Realm legend was blank. Authors may consider
  distinguishing realms with 3 more distinct line types.
\item
  In figure 5, panel b, we believe that the color should be set to
  discrete rather than continuous. Our interpretation of the description
  of these results calls for 2 shades of blue and 2 shades of green, but
  there appear to be 3+ shades of blue and green.
\item
  On line 69, ``reliable'' should be ``reliably.''
\item
  We found it extremely interesting that the overall trend in
  biodiversity loss was not significantly different than zero. It is
  absolutely fascinating that while global biodiversity is decreasing,
  local biodiversity is essentially remaining the same due to species
  turnover. We would love to hear more of the authors interpretation
  possible consequences. We are especially curious about what these
  results may mean for ecosystem function over time. Do the authors feel
  that that ecosystem function will be maintained even in an era of
  extinction due to this turnover effect?~~
\end{itemize}

It was a great pleasure reading this interesting and exciting work and
we are extremely grateful to the authors for posting it as a preprint on
bioRxiv and for their obvious commitment to open, reproducible science.
In addition to most of the data used in this paper being available in a
public data base, the code for the analyses is extremely well commented
and available on Zenodo. Meta-analyses like this, while difficult, are
absolutely necessary to illuminate larger ecological trends, and we
thoroughly appreciate the effort made here.~We sincerely hope that our
comments are useful to the authors and we look forward to reading the
final version when it is published.~

Very best wishes,

The OIST Ecology and Evolution Preprint Journal Club

\par\null

\selectlanguage{english}
\FloatBarrier
\end{document}

