\documentclass[10pt]{article}

\usepackage{fullpage}
\usepackage{setspace}
\usepackage{parskip}
\usepackage{titlesec}
\usepackage{placeins}
\usepackage{xcolor}
\usepackage{breakcites}
\usepackage{lineno}





\PassOptionsToPackage{hyphens}{url}
\usepackage[colorlinks = true,
            linkcolor = blue,
            urlcolor  = blue,
            citecolor = blue,
            anchorcolor = blue]{hyperref}
\usepackage{etoolbox}
\makeatletter
\patchcmd\@combinedblfloats{\box\@outputbox}{\unvbox\@outputbox}{}{%
  \errmessage{\noexpand\@combinedblfloats could not be patched}%
}%
\makeatother


\usepackage[round]{natbib}
\let\cite\citep




\renewenvironment{abstract}
  {{\bfseries\noindent{\abstractname}\par\nobreak}\footnotesize}
  {\bigskip}

\renewenvironment{quote}
  {\begin{tabular}{|p{13cm}}}
  {\end{tabular}}

\titlespacing{\section}{0pt}{*3}{*1}
\titlespacing{\subsection}{0pt}{*2}{*0.5}
\titlespacing{\subsubsection}{0pt}{*1.5}{0pt}


\usepackage{authblk}


\usepackage{graphicx}
\usepackage[space]{grffile}
\usepackage{latexsym}
\usepackage{textcomp}
\usepackage{longtable}
\usepackage{tabulary}
\usepackage{booktabs,array,multirow}
\usepackage{amsfonts,amsmath,amssymb}
\providecommand\citet{\cite}
\providecommand\citep{\cite}
\providecommand\citealt{\cite}
% You can conditionalize code for latexml or normal latex using this.
\newif\iflatexml\latexmlfalse
\providecommand{\tightlist}{\setlength{\itemsep}{0pt}\setlength{\parskip}{0pt}}%

\AtBeginDocument{\DeclareGraphicsExtensions{.pdf,.PDF,.eps,.EPS,.png,.PNG,.tif,.TIF,.jpg,.JPG,.jpeg,.JPEG}}

\usepackage[utf8]{inputenc}
\usepackage[english]{babel}








\begin{document}

\title{Live streamed Journal Club on ``A Measure of Open Data: A Metric and
Analysis of Reusable Data Practices in Biomedical Data Resources''}



\author[1]{Samantha Hindle}%
\author[1]{Monica Granados}%
\author[1]{Daniela Saderi, Ph.D.}%
\affil[1]{PREreview}%


\vspace{-1em}



  \date{\today}


\begingroup
\let\center\flushleft
\let\endcenter\endflushleft
\maketitle
\endgroup





\selectlanguage{english}
\begin{abstract}
As part of our first
\href{https://www.prereview.org/users/164141/articles/301936-live-prereview-journal-clubs-liveprejc}{Live
PREreview Journal Club} (\#LivePREJC), we discussed the bioRxiv preprint
``A Measure of Open Data: A Metric and Analysis of Reusable Data
Practices in Biomedical Data Resources'' by Seth
Carbon,~\href{http://orcid.org/0000-0001-7023-9832}{~}Robin Champieux,
Julie McMurry,~\href{http://orcid.org/0000-0001-7120-8536}{~}Lilly
Winfree,~\href{http://orcid.org/0000-0003-1026-5232}{~}Letisha R Wyatt,
and Melissa Haendel. doi:~\url{https://doi.org/10.1101/282830}.~ We were
joined by two of the preprint authors:~Robin Champieux and Lilly
Winfree.~ Below is a summary of our discussion.~%
\end{abstract}%




\section*{}

\section*{Summary:}

{\label{547062}}

The manuscript provided a means to increase the clarity on what is quite
a complex landscape of data licensing - to help make data re-use terms
clearer. The authors generated a clear rubric that was used to determine
how visible the licensing info was, how open the licensing terms were,
and what this means for the user in terms of data reuse restrictions,
e.g. what kind of data can be re-used and under what conditions. This
rubric provided a quantitative metric of open data that can be used more
generally in open science to answer the question ``How much of my
science/data is open?''

\par\null

The live preprint journal club~ (Live PREJC, the discussion was
conducted over video call) attracted a broad group of participants from
across the North America. Given the diverse makeup of the journal club
particpants, we were interested in why this study was relevant to their
discipline/research. Below is a summary of their responses:

\par\null

\begin{itemize}
\tightlist
\item
  I'm a theater making rapper who uses research to put together
  presentations. I'm not the only one. I~ don't have the capacity for
  review, and my counterparts just want to make~ a~ hot show, so a lot
  of mess ends up in the presentations. Also the lack of diversity
  doesn't~ help~ much either in terms of perspective in the narratives I
  make.
\end{itemize}

\begin{itemize}
\tightlist
\item
  Having been in a lab where we have used thousands of datasets from
  other labs to build a tool that can be used to analyse genomics data,
  I understand how frustrating it might be to not know what the reuse
  policies are for these datasets, and how much time it might waste
  having to request permissions. This manuscript helps to address where
  the situation lies right now, with the hope that this might help
  develop more clarity in future policies.
\end{itemize}

\begin{itemize}
\tightlist
\item
  There are many definitions of open data but they are used
  interchangeably in science its great to have a table to define the
  different types of data that get classified as open but fall at very
  different parts of that transect.
\end{itemize}

\begin{itemize}
\tightlist
\item
  I'm interested in monitoring policies across fields or different
  organizations. The development of a rubric enables comparisons, a
  birds-eye view, and advocacy for change. Furthermore, our current
  MozSprint project, TRANSPOSE, is a journal database built on a similar
  YAML architecture :) ASAPbio is also working on preprint licensing,
  and while the options typically available to people in that space are
  more restricted (to CC licenses), preprint servers offer different
  levels of clarity and machine access to these licenses. This framework
  could be extended or modified to categorize information about that
  space.
\end{itemize}

\begin{itemize}
\tightlist
\item
  I'm running Open Data repositories and I want to enable the best reuse
  of the data we collect.
\end{itemize}

\begin{itemize}
\tightlist
\item
  As an OA publisher we have an open data policy, a rubric like this
  helps us assess databases that align with our policies, as well as
  hurdles for researchers that we need to understand when putting a data
  policy in place or how it should evolve over time.
\end{itemize}

\begin{itemize}
\tightlist
\item
  I'm a graduate student and before we start performing certain
  experiments, mining publicly available databases yields very valuable
  information which can contribute greatly to our projects. However, how
  to reuse this data is not well known by everyone and publications like
  this will be helpful to educate other researchers.~
\end{itemize}

\par\null

\section*{What did the participants like about the
manuscript?}

{\label{710649}}\par\null

There was a general excitement about this work and an appreciation for
the impact it will have on the scientific community. The below are some
specific comments from the journal club participants:

\par\null

\begin{itemize}
\tightlist
\item
  Table 1 was very helpful for understanding the overall types of
  licenses.
\end{itemize}

\par\null\par\null

\section*{What could the authors improve in their
manuscript?}

{\label{234021}}

\section*{}

{\label{234021}}

Collectively, we felt that the following could help improve the clarity
of the manuscript:~

\begin{itemize}
\tightlist
\item
  As this is a more unique type of manuscript (not like a standard wet
  lab research MS \{manuscript{]}), it would be useful to have a figure
  that explains the process, for example, a flow diagram of how you went
  through the process.
\item
  I appreciated how you simplified a complex aspect of the licensing
  landscape into fairly concise categories, but it may help if you make
  the details of the criteria/rubric more clear. For example, it would
  be useful to have some short hand word for the criteria that isn't
  just a letter, so that it's easier to look at the visuals on the
  website. This was very educational, though, so thank you!
\end{itemize}

\begin{itemize}
\tightlist
\item
  Are all of the ``stars'' equal weight and importance? Curious whether
  this is intended to be used as a quantitative combined score or more
  like a suite of characteristics. If the former, are some of the
  categories considered more essential than others?
\end{itemize}

\begin{itemize}
\tightlist
\item
  I might have missed this but is there a place for the data of the
  data?~ Author's response:~yes! in the github repo :)
\end{itemize}

\begin{itemize}
\tightlist
\item
  On violations: separate out the definitions of violations. A violation
  that prevents further analysis/inclusion vs a category violation.
\end{itemize}

\begin{itemize}
\tightlist
\item
  In figure 3, what does it mean that categories B, D, and E have
  ``violations''? I was under the impression from the text under Fig 2
  that these were not scored if there are violations in A \& C
\end{itemize}

\begin{itemize}
\tightlist
\item
  I noticed that Figure 1 focuses on percentages whereas in the text you
  mainly focus on the raw numbers with percentages in parentheses. It
  might help the flow to always stick to percentages with the raw
  numbers in parentheses to give context/real numbers.
\end{itemize}

\begin{itemize}
\tightlist
\item
  The large amount of whitespace left around the figures were a bit
  distracting. Also, I agree with overlaying the numbers on the
  donut/pie chart, it was not easy on the eye. In figure 2, it would be
  helpful to mention again the scoring criteria to the figure caption so
  that if the reader forgets about the scoring criteria or they just
  happen~ to see the figure, they can understand what was the scoring
  criteria by just reading the figure caption.
\end{itemize}

\par\null

\section*{Minor comments/typos:}

{\label{294569}}

\begin{itemize}
\tightlist
\item
  I noticed in the second paragraph of the discussion there is
  repitition of the word ``that'' in the first sentence.
\end{itemize}

\begin{itemize}
\tightlist
\item
  Unlike other figure captions, caption for Fig. 3 was enclosed within a
  black frame. It stood out/ was different to the eye. I don't know if
  it is deliberate.
\end{itemize}

\par\null

\section*{Author's comments:}

{\label{991538}}\par\null

One of the novelties of Live PREJCs is that the authors can be on the
call, too. We therefore allowed 10 minutes at the end of the call to
allow the authors to comment on any of the points that were brought up
during the journal club. Below are some discussion points brought up by
Robin Champieux and Lilly Winfree (edited for clarity).

\par\null

\begin{itemize}
\tightlist
\item
  The authors wanted to add that while they intended~ it as a combined
  quantitative score, the criteria itself, hopefully, draws out the
  particular issues (and impacts) associated with each part.
\end{itemize}

\begin{itemize}
\tightlist
\item
  What stood out for the authors was aggregated data resources/data
  databases. They had early readers who found that focus confusing. They
  were hoping they had cleared that up in the MS {[}the rationale for
  choosing to focus on these specific databases{]}. But they haven't
  laid out a rationale for how the rubric and their thinking should
  apply to datasets not living in a aggregated database.~
\end{itemize}

\begin{itemize}
\tightlist
\item
  Figures: general feedback and how to move away from pie charts/donuts
\end{itemize}

\begin{itemize}
\tightlist
\item
  They appreciated our comments about the text being hard to read when
  overlaid on pie chart
\end{itemize}

\begin{itemize}
\tightlist
\item
  They were interested in ways to provide more info on what
  permissive/restrictive/copyright/left pool means-diff shades of same
  colour for similar licenses etc.
\end{itemize}

\begin{itemize}
\tightlist
\item
  They are considering making a network diagram\ldots{}a wall showing
  the barrier to data linkage, maybe just one visual example of this
\end{itemize}

\begin{itemize}
\tightlist
\item
  Are their clusters of categories of licenses etc?
\end{itemize}

\begin{itemize}
\tightlist
\item
  Many of the data resources were integrates to the Monarch project.
  Want to ask how these resources don't actually work together as they
  might intend-because of the tensions/conflicts
\end{itemize}

\begin{itemize}
\tightlist
\item
  For figures, think about how it will render in print. Having words
  within the figures is great but may be difficult to read depending on
  the color selection
\end{itemize}

\par\null\par\null

Thank you to everyone who participated in the Live PREJC, and in
particular to~Robin Champieux and Lilly Winfree for being brave first
participants as authors of the preprint. It was really valuable to have
two of the authors on the call.

\par\null

If you are interested in hosting a Live PREJC or you are a preprint
author and you would like to arrange a Live PREJC for your preprint, let
us know at
\href{mailto:contact@prereview.org}{\nolinkurl{contact@prereview.org}}~or
fill out our form
\href{https://docs.google.com/forms/d/e/1FAIpQLSdlpfxK0XEeVbUD7aHBKLf6g7rOups-uS2ZytpDKdHNBHwNZg/viewform?usp=sf_link}{here}.

\selectlanguage{english}
\FloatBarrier
\end{document}

