\documentclass[10pt]{article}

\usepackage{fullpage}
\usepackage{setspace}
\usepackage{parskip}
\usepackage{titlesec}
\usepackage{placeins}
\usepackage{xcolor}
\usepackage{breakcites}
\usepackage{lineno}





\PassOptionsToPackage{hyphens}{url}
\usepackage[colorlinks = true,
            linkcolor = blue,
            urlcolor  = blue,
            citecolor = blue,
            anchorcolor = blue]{hyperref}
\usepackage{etoolbox}
\makeatletter
\patchcmd\@combinedblfloats{\box\@outputbox}{\unvbox\@outputbox}{}{%
  \errmessage{\noexpand\@combinedblfloats could not be patched}%
}%
\makeatother


\usepackage[round]{natbib}
\let\cite\citep




\renewenvironment{abstract}
  {{\bfseries\noindent{\abstractname}\par\nobreak}\footnotesize}
  {\bigskip}

\renewenvironment{quote}
  {\begin{tabular}{|p{13cm}}}
  {\end{tabular}}

\titlespacing{\section}{0pt}{*3}{*1}
\titlespacing{\subsection}{0pt}{*2}{*0.5}
\titlespacing{\subsubsection}{0pt}{*1.5}{0pt}


\usepackage{authblk}


\usepackage{graphicx}
\usepackage[space]{grffile}
\usepackage{latexsym}
\usepackage{textcomp}
\usepackage{longtable}
\usepackage{tabulary}
\usepackage{booktabs,array,multirow}
\usepackage{amsfonts,amsmath,amssymb}
\providecommand\citet{\cite}
\providecommand\citep{\cite}
\providecommand\citealt{\cite}
% You can conditionalize code for latexml or normal latex using this.
\newif\iflatexml\latexmlfalse
\providecommand{\tightlist}{\setlength{\itemsep}{0pt}\setlength{\parskip}{0pt}}%

\AtBeginDocument{\DeclareGraphicsExtensions{.pdf,.PDF,.eps,.EPS,.png,.PNG,.tif,.TIF,.jpg,.JPG,.jpeg,.JPEG}}

\usepackage[utf8]{inputenc}
\usepackage[english]{babel}








\begin{document}

\title{BIGC - PREreview of ``Evolutionary responses to conditionality in
species interactions across environmental gradients''}



\author[1]{Ignasi Bartomeus}%
\author[2]{Oscar Godoy}%
\author[2]{carloszaragozatrello}%
\author[2]{alealonso}%
\author[2]{Miguel}%
\affil[1]{EBD-CSIC}%
\affil[2]{Affiliation not available}%


\vspace{-1em}



  \date{\today}


\begingroup
\let\center\flushleft
\let\endcenter\endflushleft
\maketitle
\endgroup





\selectlanguage{english}
\begin{abstract}
This is a preprint journal club review of\textbf{~``Evolutionary
responses to conditionality in species interactions across environmental
gradients''} by Anna M~O'Brien,~Ruairidh
J.H.~Sawers,~Jeffrey~Ross-Ibarra,~Sharon Y~Strauss.~The preprint was
originally posted on bioRxiv on December 10, 2017
(DOI:~\href{https://doi.org/10.1101/124750}{https://doi.org/10.1101/031195})~\url{https://www.biorxiv.org/content/early/2017/12/10/031195}.

\emph{Our group Biotic Interactions and Global~Change
(BIGC)}~\emph{reviewed this paper on January, 2017 .}%
\end{abstract}%




\section*{Overview and take-home
messages:}

{\label{334195}}

We selected this paper for lab-meeting because we are interested in the
conditionality of interactions. In a nutshell, conditionality means that
the effect of an interaction is not fixed, but conditional on other
parameters, like environmental conditions. This concept can be called
``context dependency'' or ``higher order effects'' depending on the
literature you read.~

\par\null

This manuscript first summarises how often the mutualistic or
antagonistic effect of species interactions on fitness is conditional to
environmental stress levels. Second, it describes theoretically how
conditionality on species interactions can promote evolutionary
adaptations or co-adaptations. Finally, it shows an experimental design
that can be used to test.

\par\null

\section*{Positive feed-back:}

{\label{495088}}

The paper was perfect to stimulate discussion in the group, as it
proposes~a clear evolutionary mechanism that is often ignored in the
literature. The idea to link conditionality to evolutionary outcomes is
very interesting and the theoretical and empirical proposed tests make
it a very comprehensive paper.~

\par\null

\section*{Major concerns:}

{\label{267722}}

Our only major concern is about the generality of this mechanisms. This
does not undermine the importance of describing a mechanism~that can
occur~under some situations, but was unclear to us how rare can be
events of evolutionary adaptations due to conditionality on
interactions.~

\par\null

\section*{Minor concerns:}

{\label{414137}}

Maybe we are not specialists in the topic but we all agree that the
paper was trying to cover a lot of ground. Maybe focusing only on
mutualisms, and providing more specific examples of interactions, stress
gradients, and characters under selection would help the reader to focus
through the paper.

\section*{}

{\label{888542}}

\section*{Other thoughts:}

{\label{146152}}

- The empirical test suggested is trying to catch evolution in the act
and won't work on species already speciated through this mechanisms. A
macroecological approach may be complementary trying to identify within
groups if species adapted to high stress also invest more in mutualists
(higher number of partners) and species adapted to low-stress
environments establish fewer interactions.

\par\null

- It is not clear how important can be the selection pressure due to
conditionality as compared to the myriad of factors (starting with the
main effects of stress) affecting a given species. Or in other words,
from the simultaneous selection pressures of mutualists, herbivores,
competitors, abiotic stressors, the species has, which is the relevance
of a single selection pressure.~

\par\null\par\null

Biotic Interactions and Global Change (BIGC) - O. Godoy and I. Bartomeus
groups.

\par\null

\selectlanguage{english}
\FloatBarrier
\end{document}

