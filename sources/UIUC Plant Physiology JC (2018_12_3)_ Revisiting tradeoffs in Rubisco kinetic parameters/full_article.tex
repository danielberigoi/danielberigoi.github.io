\documentclass[10pt]{article}

\usepackage{fullpage}
\usepackage{setspace}
\usepackage{parskip}
\usepackage{titlesec}
\usepackage{placeins}
\usepackage{xcolor}
\usepackage{breakcites}
\usepackage{lineno}





\PassOptionsToPackage{hyphens}{url}
\usepackage[colorlinks = true,
            linkcolor = blue,
            urlcolor  = blue,
            citecolor = blue,
            anchorcolor = blue]{hyperref}
\usepackage{etoolbox}
\makeatletter
\patchcmd\@combinedblfloats{\box\@outputbox}{\unvbox\@outputbox}{}{%
  \errmessage{\noexpand\@combinedblfloats could not be patched}%
}%
\makeatother


\usepackage[round]{natbib}
\let\cite\citep




\renewenvironment{abstract}
  {{\bfseries\noindent{\abstractname}\par\nobreak}\footnotesize}
  {\bigskip}

\renewenvironment{quote}
  {\begin{tabular}{|p{13cm}}}
  {\end{tabular}}

\titlespacing{\section}{0pt}{*3}{*1}
\titlespacing{\subsection}{0pt}{*2}{*0.5}
\titlespacing{\subsubsection}{0pt}{*1.5}{0pt}


\usepackage{authblk}


\usepackage{graphicx}
\usepackage[space]{grffile}
\usepackage{latexsym}
\usepackage{textcomp}
\usepackage{longtable}
\usepackage{tabulary}
\usepackage{booktabs,array,multirow}
\usepackage{amsfonts,amsmath,amssymb}
\providecommand\citet{\cite}
\providecommand\citep{\cite}
\providecommand\citealt{\cite}
% You can conditionalize code for latexml or normal latex using this.
\newif\iflatexml\latexmlfalse
\providecommand{\tightlist}{\setlength{\itemsep}{0pt}\setlength{\parskip}{0pt}}%

\AtBeginDocument{\DeclareGraphicsExtensions{.pdf,.PDF,.eps,.EPS,.png,.PNG,.tif,.TIF,.jpg,.JPG,.jpeg,.JPEG}}

\usepackage[utf8]{inputenc}
\usepackage[english]{babel}








\begin{document}

\title{UIUC Plant Physiology JC (2018/12/3): Revisiting tradeoffs in Rubisco
kinetic parameters~ ~}



\author[1]{Steven Burgess}%
\affil[1]{University of Illinois at Urbana–Champaign}%


\vspace{-1em}



  \date{\today}


\begingroup
\let\center\flushleft
\let\endcenter\endflushleft
\maketitle
\endgroup









The preprint ``Revisiting tradeoffs in Rubisco kinetic parameters'' by
Flamholz et al. 2018 (\url{https://doi.org/10.1101/470021}) investigates
the tradeoffs between catalytic efficiency and rate, of the central
enzyme in carbon fixation, Ribulose-1,6-bisphosphate
Carboxylase/Oxygenase (RuBisCO), using kinetic modeling based on
biochemical data. The manuscript builds on previous work from the group
(Savir et al. 2010; doi: 10.1073/pnas.0911663107), including an expanded
dataset of kinetic parameters~of \textasciitilde{}250 RuBisCOs from 286
different species extracted from the literature. We thought it was an
important topic with potential interest for a wide range of researchers
working on photosynthesis and evolution.

\par\null

The main questions the paper seeks to address are:

\par\null

\begin{enumerate}
\tightlist
\item
  Which trade-offs are inherent in Rubisco kinetics.
\item
  Has evolution resulted in optimal kinetics within the constraints of
  those inherent trade-offs.
\end{enumerate}

The preprint challenges the theory that increasing RuBisCO activity
reduces enzyme specificity as described by Tcherkez et al. (2006), which
was based on a model of enzyme activity the discriminates between CO2
and O2 in a transition state. Data supporting this theory has been
reported widely in the literature. However, the authors propose that
there is little/no correlation between specificity and activity, and
most previously-found correlations (KcatC and SC/O etc) are smaller for
the new dataset, except for KcatC/KC and KcatO/KO.

\par\null

We really enjoyed reading the manuscript and as it challenged our
preconceptions about RuBisCO activity. We found it interesting (and
surprising!) that the data contradicts a well-established theory, and it
increased our awareness out current models of enzyme activity. We also
thought it was interesting that they were able to collect data from so
many species across many previous studies and also break down
trends/relationships between different clades or physiologies. It was
also interesting that given that they were using data from studies that
showed the opposite, they were able to come to the conclusion they
found. ~~We particularly liked the authors' suggestions about how to
move the field forward and the call for an improved understanding of
RuBisCO kinetic mechanism.

\par\null

There were a few areas we thought it would be useful to clarify:

\begin{itemize}
\tightlist
\item
  Providing a cartoon model of the proposed mechanism would help readers
  unfamiliar with the nuances of the models being assessed.
\item
  More information about how the authors selected and filtered data
  collecting from the literature and how the different datasets were
  taken into account in the statistical analysis. i.e. how many
  measurements per species, types of values (mean/median) etc.
\item
  Figure 4 and 5: how was the conclusion reached that there was no
  correlation between parameters?
\item
  Organization, it would make things stronger to layout the arguments
  clearly between what came before and after, for those not familiar
  with the academic argument.
\item
  As the paper argues against what most people are reading it might
  expect, additional text theorizing why this is the case would be
  useful to guide readers.
\end{itemize}

Minor comments

\begin{itemize}
\tightlist
\item
  Figure 2A y-axis labels
\item
  Put a key at the top as the symbols can get buried in the text
\end{itemize}

\par\null

\textbf{}

\selectlanguage{english}
\FloatBarrier
\end{document}

